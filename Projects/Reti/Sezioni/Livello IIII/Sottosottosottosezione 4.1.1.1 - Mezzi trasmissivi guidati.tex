\documentclass{subfiles}
\begin{document}
Come anticipato sono messi in cui il segnale viaggia su di un corpo fisico; in questa categoria si trovano:
\begin{itemize}
    \item \textbf{doppini di rame:} cavi unidirezionali isolati singolarmente,
          avvolti tra loro così da reciprocamente annullare i campi elettromagnetici generati.
          Esistenti in sette categorie, le prime sei sono dette UTP (unshielded twisted pair) poiché non si procede a schermare il cavo;
          la settima categoria è detta STP (shielded twisted pair), proprio perché il cavo è schermato.

    \item \textbf{cavi coassiali:} si tratta di cavi con due conduttori concentrici in rame, la cui parte esterna fa da schermatura.
          La trasmissione può essere realizzata secondo due modalità, la prima, quella \emph{base}, consente la trasmissione di un unico canale;
          la seconda, quella \emph{larga}, permette più canali.
\end{itemize}
\end{document}