\documentclass{subfiles}
\begin{document}
Si può definire \emph{telecomunicazione} il processo di trasmissione di messaggi rta host posti a distanza.
In tale processo sono coinvolti:
\begin{itemize}
    \item \textbf{host sorgete:} invia il messaggio tramite una qualche sorgente;
    \item \textbf{un trasmettitore:} riceve il messaggio, lo codifica e lo invia;
    \item \textbf{un canale di comunicazione:} è il mezzo fisico su cui si propaga il segnale;
    \item \textbf{un ricevitore:} svolge le funzioni inverse al trasmettitore;
    \item \textbf{un destinatario:} riceve il messaggio.
\end{itemize}

In generale, un segnale consiste nella variazione dello stato fisico di un sistema o di una grandezza fisica;
la natura di questi fenomeni è tipicamente elettromagnetica.

Circa i mezzi trasmissivi, questi sono distinguibili in:
\begin{itemize}
    \item \textbf{guidati:} la trasmissione avviene tramite un corpo fisico (eg: cavi rame, fibre ottiche, ecc.);
    \item \textbf{non guidati:} la trasmissione avviene per mezzo dell'etere.
\end{itemize}

\paragraph{Mezzi trasmissivi guidati}
\subfile{../Livello IIII/Sottosottosottosezione 4.1.1.1 - Mezzi trasmissivi guidati.tex}
\end{document}