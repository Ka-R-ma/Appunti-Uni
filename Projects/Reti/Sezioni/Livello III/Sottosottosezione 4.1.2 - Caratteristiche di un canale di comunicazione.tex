\documentclass{subfiles}
\begin{document}
Grazie al fisico \emph{Jean Baptiste Joseph Fourier}, si scoprì che una qualsiasi funzione periodica di periodo \(T\),
può essere espressa come combinazione di un certo numero, eventualmente infinito, di funzioni sinusoidali.
Se ne può cioè calcolare la \emph{trasformata di Fourier}. In simboli
\begin{equation}
    g(t) = \frac{a_{0}}{2} \sum*[n = 1][\infty]{a_{n} \sin{\frac{2tn\pi}{T}} + b_{n} \cos{\frac{2tn\pi}{T}}} \\
\end{equation}
ove, posto \(\omega = 2\pi / T\), rispettivamente
\[
    a_{0} = \frac{2}{T} \int[\nicefrac{-T}{2}][\nicefrac{T}{2}]{g(t)}{t}                 \qquad
    a_{n} = \frac{2}{T} \int[\nicefrac{-T}{2}][\nicefrac{T}{2}]{g(t) \sin{\omega nt}}{t} \qquad
    b_{n} = \frac{2}{T} \int[\nicefrac{-T}{2}][\nicefrac{T}{2}]{g(t) \cos{\omega nt}}{t}
\]
\end{document}