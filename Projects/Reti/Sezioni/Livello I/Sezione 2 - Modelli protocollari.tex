\documentclass{subfiles}
\begin{document}
Un protocollo di rete deve garantire \emph{affidabilità \emph{e la possibilità di} evoluzione} dello stesso.
Nel garantire l'affidabilità, il protocollo deve prevedere meccanismi per la gestione di eventuali errori;
in genere realizzati per mezzo di codici di \emph{error detection \emph{e} error correction}.
Inoltre, affinché il protocollo sia affidabile, questi deve garantire un'assegnazione equa delle risorse.

\subsection{Stratificazione dei protocolli}
\subfile{../Livello II/Sottosezione 2.1 - Stratificazione dei protocolli.tex}

\subsection{Connessione e affidabilità}
\subfile{../Livello II/Sottosezione 2.2 - Connessione e affidabilita.tex}
\clearpage
\end{document}