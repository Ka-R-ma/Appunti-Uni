\documentclass{subfiles}
\begin{document}
Il filtro mediano tenta di risolvere il problema del rumore ``sale e pepe'', secondo la seguente logica.
Poiché il rumore sale-pepe aggiunge quà e là pixel bianchi e neri per l'immagine, da cui il nome,
l'idea è quella di considerare una porzione MxN dell'immagine, una cosiddetta finestra \(W\),
procedere ad ordinare secondo tonalità di colore i pixel nella stessa e sostituire quello in posizione centrale al pixel in posizione centrale della finestra.
Questo perché un pixel nero (analogamente uno bianco) non sarà eliminato se \(\tfrac{MN}{2} + 1\) dei pixel nella finestra sono neri.

\subfile{../../Figure/Tikz Figure/Figure 3.3 - Esempio di filtro mediano.tex}
\begin{Remark*}
    l'applicazione del filtro potrebbe non rimuovere completamente tutto il rumore.
    A tal proposito si potrebbe riapplicare il filtro aumentato le dimensioni della finestra, o applicare il filtro unicamente ai pixel soggetti a rumore.
    Ulteriore problema del filtro è il fatto che finestre troppo grandi potrebbero trasformare il filtro in uno di blur.
\end{Remark*}

\end{document}