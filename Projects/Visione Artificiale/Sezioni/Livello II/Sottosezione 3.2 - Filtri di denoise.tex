\documentclass{subfiles}
\begin{document}
Capita spesso quando si opera con immagini digitali che queste siano soggette a rumore di varia natura;
è dunque fondamentale, prima di operare con tali immagini, rimuovere quanto più rumore possibile, per farlo si utilizzano i filtri di denoise.
Come detto il rumore si può presentare in varie forme, dal semplice rumore ``sale e pepe'', si veda \emph{Figura \ref{fig:3.3.1}}, al più fastidioso rumore gaussiano.
In questa sezione si descriverà come rimuovere le due tipologie di rumore citate.

\subsubsection{Filtro mediano}
\subfile{../Livello III/Sottosottosezione 3.2.1 - Filtro mediano.tex}

\subsubsection{Filtro gaussiano}
\subfile{../Livello III/Sottosottosezione 3.2.2 - Filtro gaussiano.tex}

\end{document}