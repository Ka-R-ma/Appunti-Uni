\documentclass{subfiles}
\begin{document}
Nello spazio colore RGB ciascun colore è descritto in funzione dei suoi valori di rosso, di verde e di blu.
Ciascuno dei tre canali (componenti colore), ammette 256 sfumature distinte; segue da ciò la possibilità di rappresentate oltre 16 milioni di colori distinti.
Generalmente tra i più noti, è di norma impiegato per la memorizzazione di immagini in macchina data la sua semplicità implementativa.

\subsubsection{Spazio colore RGB: CCD e il filtro di Bayer}
\subfile{../Livello III/Sottosottosezione 2.1.1 - Spazio colore RGB: CDD e il filtro di bayer.tex}
\end{document}