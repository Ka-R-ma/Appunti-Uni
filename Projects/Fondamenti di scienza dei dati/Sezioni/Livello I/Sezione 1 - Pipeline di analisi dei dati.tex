\documentclass{subfiles}
\begin{document}
Il processo di analisi dei dati parte banalmente dall'ottenimento stesso dei dati,
cosa che può essere effettuata da database, file di log, o altre fonti.
In ogni caso quando si lavora con i dati si deve prestare attenzione a due problemi,
uno legato alla strutturazione dei dati, l'altro alla numerosità degli stessi.

In generale comunque, prima di operare con i dati si procede ad una fase di pulizia.
Questa è atta ad gestire
\begin{itemize}
    \item \emph{valori nulli o erronei:} è ovvio che la presenza di valori errati possa influire negativamente sul modello;
    \item \emph{valori duplicati:} se non gestiti potrebbero portare ad un cattivo addestramento;
    \item \emph{struttura dei dati:} è opportuno uniformare la struttura con cui i dati sono rappresentati.
\end{itemize}

Una volta effettuata la fase di pulizia, si può passare alla parte principale dell'analisi dei dati: l'estrazione delle feature.
Dal un punto di vista strettamente formale, queste rappresentano caratteristiche valorizzabili dei dati,
utili alla risoluzione di un qualche problema. Una volta estratte le feature, queste sono adattate per essere utilizzate da un modello di machine learning,
ed eventualmente combinate per definire caratteristiche supplementari.

\subsection{Integrazione dei dati}
\subfile{../Livello II/Sottosezione 1.1 - Integrazione dei dati.tex}

\subsection{Trasformazione dei dati}
\subfile{../Livello II/Sottosezione 1.2 - Trasformazione dei dati.tex}

\subsection{Riduzione della dimensionalità}
\subfile{../Livello II/Sottosezione 1.3 - Riduzione della dimensionalita.tex}
\clearpage
\end{document}