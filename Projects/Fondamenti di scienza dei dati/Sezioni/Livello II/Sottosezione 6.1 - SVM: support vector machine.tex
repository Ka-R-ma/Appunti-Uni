\documentclass{subfiles}
\begin{document}
Considerando la sua versione più elementare (il caso lineare) SVM costruisce un iper-piano della forma
$$
    \VectorBold{wx} + b = 0
$$
con $\VectorBold{w} \in \Real^{n} \text{e} b$ costante, parametrizzanti l'iper-piano,
tale da separare le osservazioni.

\begin{Remark*}
    L'iper-piano, generalmente, non è unico.
\end{Remark*}

Poiché generalmente l'iper-piano non è unico, come si stabilisce quello da utilizzare?
Di norma si tende ad utilizzare quello che massimizza il cosiddetto \emph{margine}:
si tratta della distanza tra l'osservazione più vicina all'iper-piano e il limite decisionale dello stesso.

\subsubsection{Hard margin SVM}
\subfile{../Livello III/Sottosottosezione 6.1.1 - Hard margin SVM.tex}

\subsubsection{Soft margin SVM}
\subfile{../Livello III/Sottosottosezione 6.1.2 - Soft margin SVM.tex}
\end{document}