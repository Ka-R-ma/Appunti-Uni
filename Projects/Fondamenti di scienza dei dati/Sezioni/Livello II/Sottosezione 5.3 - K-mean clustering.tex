\documentclass{subfiles}
\begin{document}
Come anticipato se $dist\left(X_{i}, Y_{j}\right) = \Norm{X_{i} - Y_{j}}[2]^{2}$, si ha a che fare con il \emph{k-mean} clustering.
Con tale algoritmo, l'utente deve solamente fornire i rappresentanti dei cluster, e da esse l'algoritmo procedera a comporli.
Nello specifico l'algoritmo procede come segue.
\begin{enumerate}
    \item per ogni elemento $x$, si calcola la distanza con ogni rappresentate \(\mu_{i}\) definito dall'utente,
          e si assegna $x$ al cluster il cui rappresentante ha la distanza minore da $x$;
    \item per ogni cluster così formato si calcola il centroide (la media);
    \item si ripete il processo finché non vi è convergenza\footnotemark[2].
\end{enumerate}

Circa la complessità dell'algoritmo, questa è \OrderOf{nkdi}, ove $n$ è il numero di elementi del dataset, $k$ il numero di cluster desiderati,
$i$ il numero di iterazioni e $d$ il numero di attributi delle osservazioni.
\end{document}