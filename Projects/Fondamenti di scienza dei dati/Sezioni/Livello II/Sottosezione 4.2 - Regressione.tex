\documentclass{subfiles}
\begin{document}
Il concetto di regressione si basa su quello di \emph{curve fitting}: assunti \(x\) i dati indipendenti,
\(y\) la variabile dipendente e \(\beta\) parametri sconosciuti, tramite funzioni semplici cerca di minimizzare l'errore di adattamento ai dati.
In generale, curve fitting è formulato come soluzione di un sistema \(Ax = b\).
\\ \\
Parlando di regressione, questa è utilizzata quando si ricerca un modello statistico che, a partire da variabili casuali \(\List{X}{1}{k}\),
determini una funzione dipendente \(Y\).
In tal senso, obbiettivo della regressione è determinare la forma della relazione funzionale tra le variabili.

Si distinguono du classi di regressione: lineare e non.

\subsubsection{Regressione lineare}
\subfile{../Livello III/Sottosottosezione 4.2.1 - Regressione lineare.tex}

\subsubsection{Regressione non lineare}
\subfile{../Livello III/Sottosottosezione 4.2.2 - Regressione non lineare.tex}
\end{document}