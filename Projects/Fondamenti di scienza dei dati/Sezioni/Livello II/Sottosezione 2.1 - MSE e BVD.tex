\documentclass{subfiles}
\begin{document}
Si è parlato di stimatori e predittori, rimane dunque da discutere come determinare la ``bonta'' degli stessi.
Sebbene ne esistano altre, la tecnica che risulta di interesse è l'\emph{MSE}.
Questi nel caso degli stimatori è definito come
$$
    MSE(\Theta_{n}) =  \Expected*{(\Theta_{n} - \Theta)^{2}} %
    = \underbrace{\Expected{\Theta_{n}}^{2} + \Theta^{2} - 2\Expected{\Theta_{n}}\Theta}_{\Bias{\Theta_{n}}^{2}} %
    + \underbrace{\Expected{\Theta_{n}^{2}} - \Expected{\Theta_{n}}^{2}}_{\Var{\Theta_{n}}}
$$
\begin{Remark*}
    Tale decomposizione prende il nome di \emph{bias-variance decomposition}.
\end{Remark*}
Supposto infine $y = f(x) + \varepsilon$ un predittore, e $\overset{\sim}{f}$ l'approssimazione di $f$ sulla base dei dati, si dimostra che
$$
    MSE(y) = \Expected*{\left(y - \overset{\sim}{f}\right)^{2}} = \cdots = \Var{f(x)} + \Var{\varepsilon} + \Bias{f(x)}^{2}
$$
\end{document}