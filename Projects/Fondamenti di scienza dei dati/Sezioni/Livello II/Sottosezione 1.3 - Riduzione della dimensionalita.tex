\documentclass{subfiles}
\begin{document}
In generale, quando si opera con i dati, questi risultano essere di grandi quantità.
Poiché gestirli tutti risulta complesso, si preferisce molto spesso ridurre al necessario i dati da utilizzare.
Per far ciò si utilizza una delle diverse tecniche di riduzione della dimensionalità, tra queste
\begin{itemize}
    \item \emph{l'aggregazione:} simile a quella descritta nella fase di trasformazione;
    \item \emph{la selezione degli attributi minimi:} si procede ad estrarre un sottoinsieme dei dati,
          tali che questi siano sufficienti a rappresentare l'intero insieme.
          Di questa tecnica si distinguono
          \begin{itemize}
              \item \emph{la selezione in avanti:} a partire dall'intero set di dati \(S\),
                    si costruisce un sottoinsieme \(S'\) estraendo di passo in passo un elemento da \(S\) e inserendolo in \(S'\),
                    se e solo se questi migliora la qualità dei dati in \(S'\).

              \item \emph{la selezione all'indietro:} da \(S\) si rimuove un elemento, se a seguito di una sua eventualmente estrazione,
                    la qualità degli elementi in \(S\) risulta massimizzata.
          \end{itemize}
\end{itemize}
\end{document}