\documentclass{subfiles}
\begin{document}
La PCA è una tecnica con funzionalità duplice, essa permette infatti di scoprire strutture a livello multidimensionale nei dati,
e permette inoltre di ridurre la dimensionalità dei dati.

Sostanzialmente, supposta $\Sigma = X^{T}X \in \Real^{n, m}$, invece di considerare tutta la matrice,
si considera la versione data dai suoi autovalori. Di questi, siano $\List{\lambda}{1}{m}$, si considera quello che ha il maggior contenuto informativo,
e fatto ciò, si procede a ridurre lungo la sua direzione.

\begin{Example*}
    sia supposto uno scenario come quello in \emph{Figura \ref{sub@Fig:1.a}}.
    \subfile{../../Figure/Tikz Figure/Figura 1 - Esempio PCA.tex}
    Eseguendo su di esso il passaggio agli autovalori (\emph{Figura \ref{sub@Fig:1.b}}),
    da cui concludendo la PCA, si ottiene \emph{Figura \ref{Fig:2}}.
    \subfile{../../Figure/Tikz Figure/Figura 2 - Effetto della PCA.tex}
\end{Example*}

Ricordando che le osservazioni della matrice possono essere visti come vettori di uno spazio $n-dimensionale$, seguono le seguenti definizioni.
\begin{Definition*}
    le nuove coordinate dei vettori corrispondenti le osservazioni, sono definiti \emph{scores}.
\end{Definition*}
\begin{Definition*}
    i pesi che definiscono le componenti principali sono detti \emph{loadings}.
\end{Definition*}
\end{document}