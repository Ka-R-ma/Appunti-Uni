\documentclass{subfiles}
\begin{document}
Sebbene esistano decine di normalizzazione diverse, di interesse risultano
\begin{itemize}
    \item \emph{la normalizzazione max-min:} si tratta di una normalizzazione dipendente dai valore di massimo e minimo dei dati.
          Nello specifico, assunti $v \text{i dati,} v' $ i dati normalizzati, $MAX_{0} \text{e} min_{0}$ rispettivamente i valori di massimo e minimo iniziali,
          e assunti inoltre $MAX \text{e} min$ gli estremi del nuovo range, segue
          $$
              v'  \frac{v - min_{0}}{(MAX_{0} - min_{0})}(MAX - min) + min
          $$
          \begin{MarginNote}
              In generale, poiché si preferisce operare con valori in \(\Range*{0}{1} \text{,} Max = 1, min = 0\).
          \end{MarginNote}

    \item \emph{la normalizzazione z-core:} è una normalizzazione dipendente dalla media e dalla deviazione standard dei dati.
          Nello specifico, posti $v \text{i dati,} v'$ i dati normalizzati, $\mu \text{la media e} \sigma$ la deviazione standard, si ha
          $$
              v' = \Frac{v - \mu}{\sigma}
          $$

    \item \emph{la normalizzazione decimale:} si normalizza dividendo per una potenza di 10 tale che,
          a seguito della normalizzazione, i valori cadano nel range $\Range{0}{1}$.
          Formalmente
          $$
              v' = \Frac{v}{10^{k}}, k = \Min{k \in \Natural}[\Norm{v'} \le 1]
          $$
\end{itemize}
\end{document}