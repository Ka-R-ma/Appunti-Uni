\documentclass{subfiles}
\begin{document}
Il concetto di personalità è qualcosa di complesso da definire.
Per tale ragione in genere gli studiosi tendono a definirla attraverso aspetti che caratterizzano la stessa.
Secondo tale ottica, la personalità è utile per
\begin{itemize}
    \item trasmettere una \emph{coerenza \emph{e} continuità} nelle azioni dell'individuo: poiché queste sono caratteristiche rilevabili nel tempo;
    \item assegnare una sorta di \emph{causalità} alle stesse.
\end{itemize}
Segue da cio che la personalità può essere utilizzata come strumento di previsione del comportamento.
Ciò dovuto all'assunzione che l'individuo si comporti coerentemente nel corso del tempo e nelle diverse situazioni.

\begin{Remark*}
    \`E importante sottolineare che la personalità non è un ente statico, viceversa essa è un'\emph{organizzazione dinamica}.
\end{Remark*}

\subsection{Oggetto di studio della psicologia della personalità}
\subfile{../Livello II/Sottosezione 1.1 - Oggetto di studio della PdP.tex}

\subsection{Teorie della personalità}
\subfile{../Livello II/Sottosezione 1.2 - Teorie della personalita.tex}

\subsection{Fattori determinanti della personalità}
\subfile{../Livello II/Sottosezione 1.3 - Fattori determinanti della personalita.tex}
\end{document}