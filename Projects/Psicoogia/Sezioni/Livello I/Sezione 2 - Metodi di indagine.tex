\documentclass{subfiles}
\begin{document}
Il processo di valutazione della personalità è noto come \emph{assessment}, durante il quale si effettuano varie valutazioni sulla base di diversi dati.
In generale, se ne distinguono quattro: i cosiddetti dati \textbf{LOTS}.
Questi, più esplicitamente, sono
\begin{itemize}
    \item \emph{(L) life record data:} si tratta di informazioni ottenibili direttamente dal vissuto del soggetto, e dunque obiettivi;
    \item \emph{(O) observer data:} dati provenienti da soggetti `terzi' in contato col soggetto.
          Questi sono distinti per il \emph{setting} (artificiale o naturale), e il tipo di osservatore;
    \item \emph{(T) test data:} di carattere puramente sperimentale, sono utilizzati per spiegare il comportamento dei soggetti sottoposti ai test;
    \item \emph{(S) self-report data:} provenienti da un'autovalutazione del soggetto.
\end{itemize}

\subsection{Metodi di analisi}
\subfile{../Livello II/Sottosezione 2.1 - Metodi di analisi.tex}

\subsection{Misure in psicologia della personalità}
\subfile{../Livello II/Sottosezione 2.2 - Misure in PdP.tex}
\clearpage
\end{document}