\documentclass{subfiles}
\begin{document}
Al modello topologico, Freud associa un modello strutturale nel quale si pone una netta distinzione tra \emph{Io, Super-Io \emph{e} Es}.
Nello specifico
\begin{itemize}
    \item l'es è associato a tutti i comportamenti istintivi e primitivi della personalità,
          risiedendo completamente nell'inconscio. Questi funge da ``motore'' per la personalità;
          da esso derivano quelli che a seguire saranno indicati come \emph{bisogni};

    \item l'io deriva dall'es e ha il compito di frenarlo limitando il soddisfacimento dei bisogni, in relazione alla realtà esterna;

    \item il super-io rappresenta l'interiorizzazione dei valori sociali e genitoriali.
          A sua volta si suddivide in
          \begin{itemize}
              \item \emph{io-ideale} cui sono associate le regole di buon comportamento;
              \item \emph{coscienza} relativo alle norme che i genitori approvano.
          \end{itemize}
\end{itemize}
\begin{Remark*}
    Il super-io agisce su tutti i livelli del modello strutturale.
\end{Remark*}
\end{document}