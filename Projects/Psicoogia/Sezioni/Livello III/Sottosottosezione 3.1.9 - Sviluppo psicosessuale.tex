\documentclass{subfiles}
\begin{document}
Nella visione Freudiana, la personalità dell'individuo si sviluppa durante l'infanzia.
Più precisamente in diverse fasi della stessa, coinvolgendo \emph{zone erogene} diverse.
Il passaggio tra queste fasi è di tipo conflittuale, conflitto che se non superato porta ad una \emph{fissazione}.
Considerando le varie fasi, queste sono
\begin{itemize}
    \item la \emph{fase orale} (0 - 18 mesi): la zona erogena è la bocca con la quale il bambino si nutre e,
          conseguentemente trae piacere.
          Si compone due sotto-fasi: \emph{incorporativa} durante la quale il bambino è completamente dipendente dall'ambiente;
          e \emph{sadica} nella quale il piacere è dato dal masticare/mordere.

    \item la \emph{fase anale} (18 mesi - 3 anni): fase durante la quale il piacere è dato dalla defecazione;
    \item la \emph{fase fallica} (3 - 5 anni): la zona erogena è l'organo genitale.
          In tale fase la libido si sposta verso il genitore di sesso opposto.
          Nascono in tal senso il \emph{complesso di Edipo\emph{e il} complesso di Elettra}.
\end{itemize}
\end{document}