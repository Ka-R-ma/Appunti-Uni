\documentclass{subfiles}
\begin{document}
La visione Freudiana di psicanalisi propone una personalità che risulta essere \emph{complessa \emph{e} dinamica},
parlando in tal senso di \emph{psicodinamica}. Nello specifico, la personalità è vista come un insieme di processi,
alcuni dei quali in opposizione tra loro, ai quali, per contrastare tale opposizione, si interpongono i \emph{meccanismi di difesa}.

\subsubsection{Modello topologico della mente}
\subfile{../Livello III/Sottosottosezione 3.1.1 - Modello topologico della mente.tex}

\subsubsection{Modello strutturale della psiche}
\subfile{../Livello III/Sottosottosezione 3.1.2 - Modello strutturale della psiche.tex}

\subsubsection{Pulsioni di vita e di morte}
\subfile{../Livello III/Sottosottosezione 3.1.3 - Pulsioni di vita e morte.tex}

\subsubsection{Repressione e rimozione}
\subfile{../Livello III/Sottosottosezione 3.1.4 - Repressione e rimozione.tex}

\subsubsection{Negazione}
\subfile{../Livello III/Sottosottosezione 3.1.5 - Negazione.tex}

\subsubsection{Proiezione}
\subfile{../Livello III/Sottosottosezione 3.1.6 - Proiezione.tex}

\subsubsection{Razionalizzazione e intellettualizzazione} Razionalizzazione e intellettualizzazione
\subfile{../Livello III/Sottosottosezione 3.1.7 - Razionalizzazione e intellettualizzazione.tex}

\subsubsection{Spostamento e sublimazione}
\subfile{../Livello III/Sottosottosezione 3.1.8 - Spostamento e sublimazione.tex}

\subsubsection{Sviluppo psicosessuale}
\subfile{../Livello III/Sottosottosezione 3.1.9 - Sviluppo psicosessuale.tex}
\end{document}