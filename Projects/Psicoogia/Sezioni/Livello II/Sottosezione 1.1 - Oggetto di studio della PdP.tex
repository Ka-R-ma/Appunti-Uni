\documentclass{subfiles}
\begin{document}
Come suggerito da \textbf{Kluckhohm} e \textbf{Murray}, ogni individuo presenta delle caratteristiche che,
per diversi aspetti, lo rendono
\begin{itemize}
    \item simile ad ogni altro essere umano (l'idea dei cosiddetti \emph{universali umani});
    \item diverso da alcuni, ma simile ad altri, per cui vi sono aspetti psicologici che permettono di ``categorizzare'' gli individui;
    \item unico da ogni altro individuo: ossia vi sono caratteristiche psicologiche proprie dell'individuo per le quali egli risulta singolare.
\end{itemize}
In breve, la psicologia della personalità (PdP) studia l'individuo nei tre aspetti citati, in aggiunta ad altri,
cercando di comprenderlo nella sua totalità.
\end{document}