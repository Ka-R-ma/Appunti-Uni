\documentclass{subfiles}
\begin{document}
Sia $\Omega$ un evento certo, sia $A$ una sua partizione, tale che $\Norm{A} > \Norm{\Natural}$.
Si può dimostrare in questi casi che si può attribuire una probabilità solamente ad un sottoinsieme $B \subset A \text{tale che} \Norm{B} = \Norm{\Natural}$.

In questi casi, se un numero aleatorio $X$ può assumere valori la cui cardinalità è pari a quella del continuo, si dice che $X$ è un numero aleatorio continuo.
In generale, se $X$ è un numero aleatorio continuo, si ha
$$
    \Prob{X = x} = 0, \forall x \in \Real
$$
Una definizione più rigorosa è la seguente.
\begin{Definition*}
    dicasi che un numero aleatorio $X$ è continuo se
    \begin{itemize}
        \item $\Prob{X = x} = 0, \forall x$;
        \item $\exists f(x) > 0$ integrabile secondo Riemann, tale che $\forall A \subseteq \Real$ misurabile secondo Peano-Jordan, si abbia
              $$
                  \Prob{A} = \Prob{X \in A} = \Int{f(x)}{x}[A]
              $$
    \end{itemize}
    \begin{Note*}
        Dicasi $f$ densità di probabilità di $X$.
    \end{Note*}
\end{Definition*}

Nel caso di numeri aleatori continui, si dimostra esservi un legame tra funzione di ripartizione e densità di probabilità.
Infatti, sia $F(x)$ la funzione di ripartizione di un qualche numero aleatorio continuo $X$, poiché per definizione
$$
    F(x) = \Prob{X \le x} = \Prob{X \in ]-\infty, x]}
$$
segue
$$
    F(x) = \Prob{X \le x} = \Int{f(t)}{t}[-\infty][x].
$$
Si ha cioè che
$$
    f(x) = F'(x)
$$

\subsection{Previsione e varianza}
\subfile{../Livello II/Sottosezione 5.1 - Previsione e varianza.tex}
\clearpage

\subsection{Distribuzione uniforme}
\subfile{../Livello II/Sottosezione 5.2 - Distribuzione uniforme.tex}

\subsection{Distribuzione esponenziale}
\subfile{../Livello II/Sottosezione 5.3 - Distribuzione esponenziale.tex}

\subsection{Distribuzione Normale}
\subfile{../Livello II/Sottosezione 5.4 - Distribuzione normale.tex}

\subsection{Distribuzione Gamma}
\subfile{../Livello II/Sottosezione 5.5 - Ditribuzione gamma.tex}
\clearpage
\end{document}