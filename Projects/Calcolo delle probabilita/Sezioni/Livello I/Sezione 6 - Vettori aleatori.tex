\documentclass{subfiles}
\begin{document}
Sia \(\Omega\) uno spazio campionario. Al generico \(\omega \in \Omega\) sono associati \(n\) valori \List{x}{1}{n},
\(n \ge 2\) che rappresentano i valori assunti da \(n\) numeri aleatori \List{X}{1}{n}.
Quest'ultimi possono essere visti come componenti di un vettore aleatorio \(X = \Tuple{X}{1}{n} \in \Real^{n}\).
Come per i singoli numeri aleatori, si distinguono il caso discreto e quello continuo.

\subsection{Vettori aleatori discreti}
\subfile{../Livello II/Sottosezione 6.1 - Vettori aleatori discreti.tex}
\clearpage

\subsection{Covarianza}
\subfile{../Livello II/Sottosezione 6.2 - Covarianza.tex}

\subsection{Matrice di varianze e covarianze}
\subfile{../Livello II/Sottosezione 6.3 - Matrice di varianza e covarianza.tex}
\clearpage

\subsection{Funzione caratteristica}
\subfile{../Livello II/Sottosezione 6.4 - Funzione caratteristica.tex}

\subsection{Convergenza}
\subfile{../Livello II/Sottosezione 6.5 - Convergenza.tex}
\end{document}