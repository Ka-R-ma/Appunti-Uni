\documentclass{subfiles}
\begin{document}
Si assuma di possedere un urna con \(N\) palline, di cui \(r\) bianche, con \(r\) non noto. E da questa effettuare \(n\) estrazioni con restituzione.
Siano ora \(\Set{\List{H}{1}{n}}\) una partizione dell'evento certo, con
\[
    H_{r} = \text{``Ci sono \(r\) palline bianche''}
\]
Sia \(E_{i} = \)``l'i-esima pallina estratta è bianca', per \(i \in \Set{1, \cdots, n}\).
Si verifica che
\[
    \Prob{E_{i}}[H_{r}] = \Frac{r}{N}   ,\qquad \forall i \in \Set{1, \cdots, n}
\]
Da ciò
\[\begin{aligned}
        \Prob{E_{i}} & = \Sum{\Prob{E_{i}}[H_{r}] \Prob{H_{r}}}{r = 0}[N] \\
                     & = \Sum{\Prob{H_{r} \Frac*{r}{N}}}{r = 0}[N]        \\
    \end{aligned}\]
Da cui risulta \(\Prob{E_{i}}\) è equiprobabile, per ogni \(i\).

Sia ora posto \(X\) il numero aleatorio di palline estratte alle \(n\) prove. Si ha
\[
    \Prob{X = h}[H_{r}]  = \binom{n}{h} \left(\Frac{r}{N}\right)^{h} \left(\Frac{N - h}{N}\right)^{n - h}                              \\
\]
da cui
\begin{equation}
    \Prob{X = h} = \Sum{\binom{n}{h} \left(\Frac{r}{N}\right)^{h} \left(\Frac{N - h}{N}\right)^{n - h} \Prob{H_{r}}}{r = 0}[N]
\end{equation}

Dicasi che \(X\) è una mistura di binomiali.
\begin{Note*}
    nel caso si effettuino estrazioni senza restituzione, la \eqref{Eq:1} diventa
    \[
        \Prob{X = h} = \Sum{\Frac{\binom{n}{h} \left(\Frac{r}{N}\right) \left(\Frac{N - h}{N}\right)}{\binom{N}{r}}\Prob{H_{r}}}{r = 0}[N]
    \]
    Si dirà in questo caso che \(X\) è una mistura di iper-geometriche.
\end{Note*}
\end{document}