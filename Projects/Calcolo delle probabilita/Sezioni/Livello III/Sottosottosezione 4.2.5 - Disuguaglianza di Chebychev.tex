\documentclass{subfiles}
\begin{document}
\begin{Theorem}
    Dato un numero aleatorio discreto \(X\) con \(\Expected{X} = \mu < \infty \text{e} \Var{X} < \infty\),
    si ha
    \[
        \Prob{\Abs{X - \mu} \ge \varepsilon} \le \Frac{\sigma^{2}}{\varepsilon^{2}}, \forall \varepsilon > 0
    \]
    \begin{Proof*}
        a partire da Markov (\emph{Teorema \ref{Thm:4.1}}), si consideri \(Y = (X - \mu)^{2}\).\\
        Si ha che \(Y \ge 0 \text{e} \Expected{Y} = \sigma^{2}\). Da Markov
        \[
            \Prob{Y \ge \alpha} \le \frac{\Expected{Y}}{\alpha}
        \]
        posto allora \(\alpha = \varepsilon^{2}\), segue
        \[
            \Prob{(X - \mu)^{2} \ge \varepsilon^{2}} \le \Frac{\sigma^{2}}{\varepsilon^{2}}
        \]
        ma poiché \((X - \mu)^{2} \ge \varepsilon^{2} \iff \Abs{X - \mu} \ge \varepsilon\), segue
        \[
            \Prob{\Abs{X - \mu} \ge \varepsilon} \le \Frac{\sigma^{2}}{\varepsilon^{2}}
        \]
    \end{Proof*}
\end{Theorem}
\clearpage
\end{document}