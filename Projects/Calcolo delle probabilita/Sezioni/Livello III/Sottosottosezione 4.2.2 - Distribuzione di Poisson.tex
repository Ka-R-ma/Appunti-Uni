\documentclass{subfiles}
\begin{document}
Sia \(X\) un numero aleatorio discreto, questi si dice avere distribuzione di Poisson se
\begin{enumerate}
    \item \(X \in \Natural_{0} \implies X \in \Set{0, 1, \cdots, n, \cdots}\)
    \item \(\Prob{X = n} = p_{n} = \tfrac{\lambda^{n}}{n!} e^{-\lambda}, \forall n \in \Natural_{0}\)
\end{enumerate}
Si dimostra che
\[
    \Expected{X} = \Var{X} = \lambda
\]
\begin{Note*}
    Tale distribuzione permette di approssimare una binomiale.
\end{Note*}

\paragraph{Proprietà di assenza di memoria}
\subfile{../Livello IIII/Sottosottosottosezione 4.2.2.1 - Proprieta di assenza di memoria.tex}
\end{document}