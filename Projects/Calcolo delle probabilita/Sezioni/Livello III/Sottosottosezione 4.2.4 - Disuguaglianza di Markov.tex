\documentclass{subfiles}
\begin{document}
\begin{Theorem}
    Dato un numero aleatorio discreto non-negativo \(X\) con \(\Expected{X} < \infty\) e un \(\alpha > 0 \in \Real\), si ha
    \[
        \Prob{X \ge \alpha} = \Frac{\Expected{X}}{\alpha}
    \]
    \begin{Proof*}
        poiché \(X \ge 0\), segue che \(\Prob{X \ge 0} = 1\). Posto ora \(\Prob{X \ge x_{n}} = p_{n}\), segue
        \[\begin{aligned}
                \alpha \Prob{X \ge \alpha} & = \alpha \Sum{p_{n}}{x_{n} \Such x_{n} \ge \alpha} \\
                                           & \le \Sum{x_{n}p_{n}}{x_{n} \Such x_{n} \ge \alpha} \\
                                           & \le \Sum{x_{n} P_{n}}{x_{n}} = \Expected{X}
            \end{aligned}\]
        da cui dunque
        \[
            \Prob{X \ge \alpha} = \Frac{\Expected{X}}{\alpha}
        \]
    \end{Proof*}
\end{Theorem}
\end{document}