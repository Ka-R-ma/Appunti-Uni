\documentclass{subfiles}
\begin{document}
Si assuma di possedere un urna con \(N\) palline, di cui \(pN\) bianche e \(qN\) nere sono note, con \(qN + pN = N\).
E da questa effettuare \(n\) estrazioni con restituzione.
Sia posto \(E_{i}\) l'evento ``l'i-esima pallina estratta è bianca' e sia
\[
    X = \Sum{\Abs{E_{i}}}{i = 1}[n].
\]
Si ha che
\[
    \Prob{E_i} = \Frac{\binom{N - 1}{pN - 1}}{\binom{N}{pN}} = p
\]
Cioè gli eventi sono equiprobabili.

Sia ora considerato il generico costituente di \(X\), si ha
\[
    \Prob{E_{i_{1}}E_{i_{2}} \cdots E_{i_{h}}E^{C}_{i_{h + 1}} \cdots E^{C}_{i_{n}}} = \cdots = \Frac{\binom{N - n}{pN - h}}{\binom{N}{pN}}
\]
da cui dunque
\[
    (X = h) = \Bigvee{E_{i_{1}}E_{i_{2}} \cdots E_{i_{h}}E^{C}_{i_{h + 1}} \cdots E^{C}_{i_{n}}}{\Set{i_{1}, \cdots, i_{h}} \subseteq \Set{1, \cdots, n}}
\]
implicando che
\[
    \Prob{X = h} = \Frac{\binom{N - n}{pN - h} \binom{n}{h}}{\binom{N}{pN}}
\]
poiché esistono \(\tbinom{n}{h}\) modi per ottenere \((X = h)\).
\end{document}