\documentclass{subfiles}
\begin{document}
\begin{Theorem*}[centrale del limite]
    Sia \(\Sequence{X_{n}}\) una successione di numeri aleatori indipendenti ed equi-distribuite, con \(\Expected{X_{i}} < \infty, \Var{X_{i}} < \infty\).
    Sia per ogni \(n \in \Natural\)
    \[
        Z_{n} = \Frac{\Sum{X_{i} - n\mu}{i = 1}[n]}{\sqrt{\sigma^{2} n}} = \Frac{1}{\sqrt{n}} \Sum{\Frac{X - \mu}{\sigma}}{i = 1}[n]
    \]
    la media aritmetica di \List{X}{1}{n}.
    Allora, la successione \(\Sequence{Z_{n}}\) converge ad un numero aleatorio \(Z\) con distribuzione normale standard.
    Ossia
    \[
        \Lim{n}{\infty}{\Prob{Z_{n} \le z}} = \Int{\Frac{1}{\sqrt{2\pi}} e^{-\Frac{t^{2}}{2}}}{t}[-\infty][z], \forall z \in \Real
    \]
    \begin{Proof*}
        sia \(\varphi_{n}(t) = \Expected{e^{itZ_{n}}}, n \in \Natural\), si dimostra banalmente che
        \[
            \varphi_{n}(t) \to e^{-\Frac{t^{2}}{2}}
        \]

        Sia ora \(Y_{i} = \Frac*{(X_{i} - \mu)}{\sigma}\), cioè \(X_{i}\) standardizzato, si ha
        \[
            Z_{n} = \Frac{1}{\sqrt{n}} \Sum{Y_{i}}{i = 1}[n]
        \]
        Considerando la funzione caratteristica di \(Z_{n}\), segue
        \[\begin{aligned}
                \varphi_{n}(t) & = \varphi_{Y_{1}}(\Frac*{t}{\sqrt{n}}) \varphi_{Y_{2}}(\Frac*{t}{\sqrt{n}}) \cdots \varphi_{Y_{n}}(\Frac*{t}{\sqrt{n}}) \\
                               & = \left[ \varphi_{Y_{i}}(\Frac*{t}{\sqrt{n}}) \right]^{n}
            \end{aligned}\]
        dove l'ultima uguaglianza deriva dal fatto che gli \(Y_{i}\) sono stocasticamente indipendenti.
        Dalla relazione tra i momenti e le derivate calcolate in zero, si ha \(\varphi_{X}^{(k)} = i^{k} \Expected{X^{n}}\).
        Da ciò, considerata la serie di MacLaurin arrestata al secondo ordine, si ha
        \[
            \varphi_{Y_{1}} = \varphi_{Y_{1}}(0) + \varphi_{Y_{1}}'(0)t + \varphi_{Y_{1}}''(0) \Frac{t^{2}}{2!} + \LittleO{t^{2}}
        \]
        da cui
        \[\begin{aligned}
                \Lim{n}{\infty}{\varphi_{n}(t)} & = \Lim{n}{\infty}{\left(1 - \Frac{t^{2}}{2n} + \LittleO{\Frac{t^{2}}{n}}\right)^{n}} \\
                                                & = e^{\Lim{n}{\infty}{n \Log*{1 - \Frac{t^{2}}{2n} + \LittleO{\Frac{t^{2}}{n}}}}}
            \end{aligned}\]
        Poiché \(\Lim{x}{0}{\Log*{\Frac*{x + 1}{x}}} = 1\), segue
        \[
            \Lim{n}{\infty}{n \Log*{1 - \Frac{t^{2}}{2n} + \LittleO{\Frac{t^{2}}{n}}}} = \cdots = \Lim{n}{\infty}{n \Log*{1 - \Frac{t^{2}}{2n}}}
        \]
        In conclusione, da ciò
        \[
            \Lim{n}{\infty}{\varphi_{n}(t)} = \Lim{n}{\infty}{n \Log*{1 - \Frac{t^{2}}{2n}}} = e^{-\Frac{t^{2}}{2}}
        \]
    \end{Proof*}
\end{Theorem*}
\begin{Remark*}
    Siano \List{E}{1}{n} una successione di eventi indipendenti ed equiprobabili, con \(\Prob{E_{i}} = p, \text{e sia} X_{i} = \Abs{E_{i}} \forall i\).
    Si ha \(\Expected{X_{i}} = p, \Var{X_{i}} = p(1 - p)\).
    Posto, per ogni \(n \in \Natural\), che
    \[
        S_{n} = X_{1} + \cdots + X_{n} = \Abs{E_{1}} + \cdots +\Abs{X_{n}}
    \]
    si ha che \(S_{n} \sim Bin(n, p), \text{con} \Expected{S_{n}} = np \text{e} \Var{S_{n}} = np(1 - p)\)
    per il teorema centrale del limite, segue
    \[
        \Prob{\Frac{S_{n} - np}{\sqrt{np(1-p)}} \le x} \xrightarrow[n \to \infty]{} \Phi_{0, 1}(x)
    \]
    Ossia, per \(n\) grandi, la distribuzione binomiale può essere approssimata da una normale standard.
\end{Remark*}
\end{document}