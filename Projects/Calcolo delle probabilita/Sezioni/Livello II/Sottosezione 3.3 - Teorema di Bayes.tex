\documentclass{subfiles}
\begin{document}
Dalla formula di disintegrazione segue il seguente risultato, noto come \emph{Teorema di Bayes}.
\begin{Theorem*}[di Bayes]
    Siano \(E, H\) due eventi. Allora
    \[
        \Prob{H}[E] = \Frac{\Prob{H} \Prob{E}[H]}{\Prob{E}}
    \]
\end{Theorem*}
Questi, nella sua versione più generale stabilisce quanto segue.
\begin{Theorem*}[di Bayes (generalizzato)]
    Siano \(\Set{\List{H}{1}{n}}\) un partizione di \(\Omega\), sia \(E\) un evento.
    Si ha allora
    \[
        \Prob{H_{i}}[E] = \Frac{\Prob{H_{i}} \Prob{E}[H_{i}]}{\Prob{E}[H_{1}] \Prob{H_{1}} + \cdots + \Prob{E}[H_{n}] \Prob{H_{n}}}
    \]
\end{Theorem*}

\subsubsection{Indipendenza stocastica}
\subfile{../Livello III/Sottosottosezione 3.3.1 - Indipendenza stocastica.tex}
\end{document}