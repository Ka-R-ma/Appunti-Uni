\documentclass{subfiles}
\begin{document}
Come evidente dall'esempio di De Mere, vi è una naturale tendenza ad associare la probabilità di un evento alla frequenza con cui lo stesso avviene.
Tale concezione di probabilità è detta \emph{impostazione frequentista}.
\begin{Definition*}
    considerata una successione di prove indipendenti, ripetute sotto le stesse condizioni,
    posto $E$ un evento e $f_{N}$ la frequenza di successo entro le prime $N$ prove, si pone
    $$
        \Prob{E} = \Lim{N}{\infty}{f_{N}}
    $$
\end{Definition*}
\end{document}