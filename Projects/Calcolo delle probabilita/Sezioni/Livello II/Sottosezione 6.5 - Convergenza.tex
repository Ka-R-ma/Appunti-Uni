\documentclass{subfiles}
\begin{document}
Sia \(\Sequence{X_{n}}\) una successione di numeri aleatori,
la \emph{convergenza} cerca di stabilire se tale successione possa convergere, in termini probabilistici,
ad un qualche numero aleatorio \(X\) con una qualche distribuzione di probabilità.
Cioè, considerato \(B \subset \Real\), (in genere si considera la classe Borel \(\Borel \subset \Real\)), si deve verificare
\[
    \Lim{n}{\infty}{\Prob{X_{n} \in B}} = \Prob{X \in B}, \forall B \subset \Borel
\]
Inoltre, dato il legame tra funzione di ripartizione e numeri aleatori,
se \(\Sequence{F_{n}}\) è una successione di funzioni di ripartizione di una successione di numeri aleatori \(\Sequence{X_{n}}\), allora
\[
    \Lim{n}{\infty}{\Prob{X_{n} \in B}} = \Prob{X \in B}, \forall B \subset \Borel   \iff \Lim{n}{\infty}{F_{n}(x)} = F(x), \forall B \subset \Borel, x \in \Real
\]
con \(F(x)\) funzione di ripartizione del numero aleatori con distribuzione limite.
\begin{Definition*}
    una successione di funzioni di ripartizione \(\Sequence{F_{n}}\) convergere a una distribuzione limite se
    \[
        \exists F_{X}(x) \Such \Lim{n}{\infty}{F_{n}(x)} = F_{X}(x), \text{per ogni punto di} F_{X}
    \]
    Si scriverà \(F_{n} \to F_{X}\)
\end{Definition*}

\begin{Theorem*}[di convergenza in distribuzione e funzione caratteristica]
    Sia \(\varphi\) funzione caratteristica di \(F_{X}\), allora la successione di funzioni di ripartizione \(\Sequence{F_{n}}\) converge ad \(F_{X}\)
    se e solo se la successione \(\Sequence{\varphi_{n}}\) di funzioni caratteristiche corrispondenti tendono a \(\varphi\). Cioè
    \[
        F_{n} \to F_{X} \iff \varphi_{n} \to \varphi
    \]
\end{Theorem*}
\begin{Note*}
    Questo risultato, permette la dimostrazione del teorema centrale del limite.
\end{Note*}
\clearpage

\subsubsection{Teorema centrale del limite}
\subfile{../Livello III/Sottosottosezione 6.5.1 - Teorema centrale del limite.tex}
\end{document}