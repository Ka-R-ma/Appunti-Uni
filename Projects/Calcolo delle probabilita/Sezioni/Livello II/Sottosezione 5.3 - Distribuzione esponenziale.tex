\documentclass{subfiles}
\begin{document}
Sia \(X\) un numero aleatorio continuo, e si la sua densità di probabilità definita come
\[
    f(x) = \begin{cases}
        \lambda e^{-\lambda x}, \lambda \in \Real^{+}, x > 0 \\
        0, x \le 0
    \end{cases}\]
Segue dalla definizione, che
\[
    F(x) = \begin{cases}
        \Int{\lambda e^{-\lambda x}}{x}[0][x], \text{se} x > 0 \\
        0, \text{se} x \le 0
    \end{cases}   \implies \begin{cases}
        1 - e^{-\lambda x}, \text{se} x > 0 \\
        0, \text{se} x \le 0
    \end{cases}\]
Dalla definizione generale di valore atteso e varianza, segue
\[\begin{aligned}
        \Expected{X} & = \Int{x \lambda e^{-\lambda x}}{x}[-\infty][\infty] = \cdots = \Frac{1}{\lambda}             \\
        \Var{X}      & = \Int{(x - \mu) \lambda e^{-\lambda x}}{x}[-\infty][\infty] = \cdots = \Frac{1}{\lambda^{2}}
    \end{aligned}\]
Vale in fine la seguente proprietà.
\begin{Property*}
    Sia \(X\) un numero aleatorio con distribuzione esponenziale, allora
    \[
        \Expected{X} = \Frac{n!}{\lambda^{n}}
    \]
\end{Property*}
\end{document}