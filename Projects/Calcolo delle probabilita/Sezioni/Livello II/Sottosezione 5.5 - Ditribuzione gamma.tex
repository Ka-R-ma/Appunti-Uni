\documentclass{subfiles}
\begin{document}
Sia \(X\) un numero aleatorio continuo, sia la sua distribuzione di probabilità definita come
\[
    f(x) = \begin{cases}
        \Frac{\lambda^{a}}{\Gamma(a)}x^{a - 1}e^{-\lambda x}, x > 0 \\
        0, x \le 0
    \end{cases}\]
allora il suo valore atteso è definito come
\[\begin{aligned}
        \Expected{X} & = \Int{x \Frac{\lambda^{a}}{\Gamma(a)}x^{a - 1}e^{-\lambda x}}{x}[0][\infty] \\
                     & = \cdots = \Frac{a}{\lambda}
    \end{aligned}\]
Poiché si dimostra che
\[
    \Expected{X^{k}} = \Frac{(a + k - 1)(a + k)\cdots a}{\lambda^{k}}
\]
segue che \(\Var{X} = \Frac*{a}{\lambda^{2}}\)

\begin{Note*}
    se \(\lambda = \Frac*{1}{2}\) e la densità di probabilità di \(X\) è definita come
    \[
        f(x) = \begin{cases}
            \Frac{1}{2^{\Frac{n}{2}} \Gamma(\Frac{n}{2})} x^{\Frac{n}{2} - 1} e^{-\Frac{x}{2}}, x \ge 0 \\
            0, x < 0
        \end{cases}\]
    si dirà che \(X\) ha distribuzione \emph{Chi-Quadro con n gradi di libertà}.
\end{Note*}
\end{document}