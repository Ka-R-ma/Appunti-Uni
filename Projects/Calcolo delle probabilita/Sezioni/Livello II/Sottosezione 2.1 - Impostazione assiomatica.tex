\documentclass{subfiles}
\begin{document}
L'impostazione assiomatica da una definizione rigorosamente matematica di probabilità, a seguito riportata.
\begin{Definition*}
    dato \((\Omega, A)\) uno spazio di misura: ossia \(\Omega\) è uno spazio con un'algebra (o una \(\sigma-algebra \text{e} A\)) una sua sottofamiglia di eventi;
    dicasi che una funzione \(\Pr \Such A \to \Range*{0}{1}\), dicasi probabilità se:
    \begin{enumerate}
        \item \(\forall E \in A, \Prob{E} \ge 0\);
        \item \(\Prob{\Omega} = 1\);
        \item \(\Prob{E_{1} \lor E_{2}} = \Prob{E_{1}} + \Prob{E_{2}}, \forall E_{1}, E_{2} \in A \Such E_{1}E_{2} = \varnothing\)
    \end{enumerate}
    Nel caso in cui \(A\) sia infinito, il punto 3 diventa
    \[
        \Prob{\Bigvee{E_{i}}{i = 1}[\infty]} = \Sum{\Prob{E_{i}}}{i = 1}[\infty]
    \]
\end{Definition*}
\end{document}