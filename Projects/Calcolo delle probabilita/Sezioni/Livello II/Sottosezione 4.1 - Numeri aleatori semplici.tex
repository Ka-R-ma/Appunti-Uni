\documentclass{subfiles}
\begin{document}
Siano \(\List{E}{1}{n}\) e siano \(\List{\alpha}{1}{n}\) numeri reali.
Si dice che
\[
    X = \alpha_{1}\Abs{E_{1}} + \cdots + \alpha_{n}\Abs{E_{n}}
\]
è un numero aleatorio semplice.

Per come è costruito, il dominio di \(X\) non è definito ma lo è il codominio,
codominio che può essere stabilito considerando i costituenti dell'evento.
Nello specifico, se tutti gli eventi sono indipendenti, allora \(X\) può assumere \(2^{n}\) valori.

\subsubsection{Distribuzione ipergeometrica}
\subfile{../Livello III/Sottosottosezione 4.1.1 - Distribuzione ipergeometrica.tex}

\subsubsection{Distribuzione binomiale}
\subfile{../Livello III/Sottosottosezione 4.1.2 - Distribuzione binomiale.tex}

\subsubsection{Mistura di binomiali}
\subfile{../Livello III/Sottosottosezione 4.1.3 - Mistura di binomiali.tex}
\end{document}