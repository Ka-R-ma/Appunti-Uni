\documentclass{subfiles}
\begin{document}
La funzione caratteristica è uno strumento teorico utile ad analizzare diversi aspetti dei numeri aleatori.
Nello specifico, questa è definita come segue.
\begin{Definition*}
    sia $X$ un numero aleatorio, e sia
    $$
        Y = e^{itX} = \Cos{tX} + i\Sin{tX}
    $$
    si definisce funzione caratteristica $\varphi_{X}$ quanto segue.
    $$\begin{cases}
            \Sum{p_{h} e^{itx_{h}}}{h}, \text{se} X \text{è discreto} \\
            \Int{e^{itx}}{x}[-\infty][\infty], \text{se} X \text{è continuo}
        \end{cases}$$
\end{Definition*}

Considerando unicamente il caso continuo, vale lo stesso ragionamento nel caso discreto, valgono le seguenti proprietà:
\begin{enumerate}
    \item $\varphi_{X}(0) = 1$
    \item $\Abs{\varphi_{X}(t)} \le \varphi_{X}(0)$, si ha infatti
          $$\begin{aligned}
                  \Abs{\varphi_{X}(t)} = \Abs{\Int{e^{itx} f(x)}{x}[-\infty][\infty]} & \le \Int{\Abs{e^{itx} f(x)}}{x}[-\infty][\infty]   \\
                                                                                      & = \Int{\Abs{e^{itx}} f(x)}{x}[-\infty][\infty] = 1
              \end{aligned}$$
\end{enumerate}

\subsubsection{Somma di numeri aleatori stocasticamente indipendenti}
\subfile{../Livello III/Sottosottosezione 6.4.1 - Somma di numeri aleatori stocasticamente indipendenti.tex}
\end{document}