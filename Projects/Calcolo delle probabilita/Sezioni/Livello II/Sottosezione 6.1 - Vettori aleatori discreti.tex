\documentclass{subfiles}
\begin{document}
\begin{Definition*}
    sia $X = \Tuple{X}{1}{n}$ un vettore aleatorio.
    Questi si dice essere discreto se
    $$
        \exists C \subset \Real^{n} \Such \forall x \in C, \Prob{X = x} \ge 0 \land \Prob{X = y} = 0 \forall y \notin C
    $$\vspace{-10pt}
\end{Definition*}

Sia $n = 2$, e siano $X, Y$ due numeri aleatori. Si indica con
$$
    C_{(X, Y)} = \Set{(x_{i}, y_{j}) \in C \subset \Real^{2}}[\Prob{X = x_{i}, Y = y_{j}} = p_{x_{i}, y_{j}} > 0]
$$
Risulta che $C \subseteq C_{X} \times C_{Y}$.

Sia ora fissato un $x_{i} \in C_{X}$, osservando che
$$
    \Omega = \Bigvee*{Y = y_{j}}{y_{j} \in C_{Y}}
$$
si può decomporre $(X = x_{i})$ come
$$\begin{aligned}
        (X = x_{i}) & = (X = x_{i}) \land \Omega                                 \\
                    & = \cdots = \Bigvee*{X = x_{i}, Y = y_{j}}{y_{j} \in C_{Y}}
    \end{aligned}$$
da cui
\begin{equation}
    \Prob{X = x_{i}, Y = y_{j}} = \Sum{p_{x_{i}, y_{j}}}{y_{j}}
\end{equation}
dicasi la $\eqref{Eq:2}$, distribuzione marginale di $X$.
Se
$$
    \Prob{X = x_{i}, Y = y_{j}} = \Prob{Y = y_{j}}[X = x_{i}] \Prob{X = x_{i}}
$$
si parla di distribuzione marginale condizionata.

\subsubsection{Indipendenza stocastica}
\subfile{../Livello III/Sottosottosezione 6.1.1 - Indipendenza stocastica.tex}
\end{document}