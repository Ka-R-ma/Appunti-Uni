\documentclass{subfiles}
\begin{document}
\begin{Definition*}
    Sia \(G\) un grafo non orientato. Si definisce ciclo euclideo un ciclo \(C\) in \(G\) tale che,
    \(C\) contenga tutti gli archi di \(G\) una e una sola volta.
\end{Definition*}

Rimane il problema di come determinare se un grafo abbia o meno un ciclo euclideo.
Fortunatamente, il seguente teorema fornisce una condizione necessaria e sufficiente affinche un grafo ammetta un ciclo euclideo.

\begin{Theorem}
    Sia \(G\) un grafo non orientato, allora \(G\) ammette un ciclo euclideo se e solo se \(G\) è connesso e ogni suo nodo ha grado pari.
    \begin{MarginNote}
        Si dimostra che, con le dovute modifiche,
        un grafo orientato ha un ciclo euclideo se per ogni nodo il numero di archi entrati è uguale a quello degli archi uscenti da esso.
    \end{MarginNote}

    \begin{Proof*}
        la dimostrazione segue dai seguenti lemmi.

        \begin{Lemma}
            Se \(G\) ha un ciclo euclideo, allora ogni nodo di \(G\) ha grado pari.

            \begin{Proof*}
                sia \(G = (V, E)\) un grafo con ciclo euclideo \(C\). Scelto un qualsiasi nodo \(v \in C\),
                necessariamente si passera da esso un numero pari di volte. Segue che \(deg(v) = 2h, h \in \Natural\).
            \end{Proof*}
        \end{Lemma}

        \begin{Lemma}
            Sia \(G\) un grado non diretto, connesso e tale che ogni nodo abbia grado pari.
            Supposto \(P\) un cammino in \(G\) tale che questi sia privo di cicli, a meno di ripetizioni di archi,
            allora questi può essere esteso ad un cammino \(P'\) più lungo.

            \begin{Proof*}
                Poiché non è un ciclo, \(P\) inizia e finisce in nodi distinti. Sia allora \(t\) il nodo terminale.
                Essendo che \(P\) termina in \(t\), segue che ogni qualvolta passa per esso, o termina o lo lascia per poi ritornarvi successivamente.
                Contando dunque il numero di archi in \(P\) incidenti a \(t\), segue che tale numero è dispari, ma si è assunto che ogni nodo abbia grado pari;
                esiste pertanto un nodo \((t, u) \notin P\). Da ciò segue che è possibile estendere \(P \text{a} P'\) aggiungendo semplicemente tale arco.
            \end{Proof*}
        \end{Lemma}

        \begin{Lemma}
            Sia \(G\) un grafo non diretto e connesso, tale che ogni nodo abbia grado pari.
            Se \(C\) è un ciclo in \(G\) che non include tutti gli archi, allora è possibile estendere \(C \text{a un cammino} P'\) che include tutti i suoi archi.

            \begin{Proof*}
                sia \(M\) l'insieme dei nodi in \(C\). Ciò che si deve dimostrare è che esiste \((u, v) \in E \setminus C \Such u \in M\).
                Dimostrata l'esistenza di tale arco, l'estensione a \(P'\) è garantita per il \emph{Lemma \ref{Lemma:4.3.2}}.
                Si distinguono due possibili casi:
                \begin{itemize}
                    \item esistono degli archi non utilizzati, per cui si può immediatamente costruire \(P'\);

                    \item non vale il caso precedente: si considera allora un qualsiasi \(x \in M\) e \(y \in V \setminus M\).
                          Poiché \(G\) è connesso, esiste un cammino tra \(x \text{e} y\), sia questi \(P\).
                          Sia ora \(u\) l'ultimo nodo di \(P\) che sia anche in \(M\). Essendo che \(P\) termina in un nodo in \(V \setminus M\),
                          \(u\) non può essere l'ultimo nodo di \(P\); sia allora \(v\) il nodo successivo a \(u \text{in} P\).
                          Per quanto detto, \(u\) è l'ultimo nodo in \(P\) che sia in \(M\), segue allora che \(v \notin M\).
                          Esiste allora \((u,v) \notin C\).
                \end{itemize}
            \end{Proof*}
        \end{Lemma}
    \end{Proof*}
\end{Theorem}

\end{document}