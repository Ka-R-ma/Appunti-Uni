\documentclass{subfiles}
\begin{document}
Dato un grafo \(G\) non diretto, si dirà che esso ha un ciclo hamiltoniano se e solo se,
in esso esiste un ciclo tale per cui questi attraversa ogni nodo del grafo una ed una sola volta.
\\ \\
Differentemente da un ciclo euleriano, per cui è possibile definire un algoritmo che verifichi la sua esistenza all'interno di un grafo \(G\),
nel caso dei cicli hamiltoniani, la ricerca degli stessi si dimostra notevolmente più complessa.
Si dimostra infatti che essa rientri nella classe dei problemi \(\mathcal{NP}-completi\).
\end{document}