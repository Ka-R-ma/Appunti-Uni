\documentclass{subfiles}
\begin{document}
\begin{Definition*}
    sia \(M = 2h, h > 0\), si definisce B-Tree di ordine \(M\) un albero con \(k\) nodi interni o un k-nodo,
    tali che
    \begin{itemize}
        \item ogni cammino radice-nodo esterno ha la stessa lunghezza;
        \item per la radice \(2 \le k \le M -1\);
        \item per ogni altro nodo \(\nicefrac{M}{2} \le k \le M -1\).
    \end{itemize}

    \begin{Remark*}
        Si può pensare ai B-Tree come ad una generalizzazione degli alberi 2-3.
    \end{Remark*}
\end{Definition*}
\begin{Example*}
    sia \(M = 6\). Un B-Tree di ordine 6 è quello in \emph{Figura \ref{sub@Fig:4}}
    \subfile{../../Figure/Tikz Figure/Figure 4 - Esempio BTree ordine 6.tex}
\end{Example*}

Circa le operazioni, si dimostra che, dato un B-Tree di ordine \(M\) con \(N\) chiavi,
queste richiedono un tempo compreso tra \(\OrderOf{\Log{N}[M - 1]} \text{e} \OrderOf{\Log{N}[\nicefrac{M}{2}]}\).

\begin{Remark*}
    I B-Tree sono generalmente utilizzati per la gestione della memoria. Nello specifico per operazioni di I/O.
    Segue da questo che scegliendo blocchi proporzionali alla dimensione dei blocchi di memoria \(B\),
    la complessita si riduce a \(\OrderOf{\Log{N}[2]}\).
\end{Remark*}

\subsubsection{Ricerca}
\subfile{../Livello III/Sottosottosezione 1.5.1 - Ricerca in un B-Tree.tex}

\subsubsection{Inserimento}
\subfile{../Livello III/Sottosottosezione 1.5.2 - Inserimento in un B-Tree.tex}
\end{document}