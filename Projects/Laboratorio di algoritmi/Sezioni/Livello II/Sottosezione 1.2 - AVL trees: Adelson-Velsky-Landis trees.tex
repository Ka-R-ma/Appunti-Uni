\documentclass{subfiles}
\begin{document}
Gli AVL sono una tipologia di alberi binari di ricerca bilanciati in altezza.
Nello specifico, si dice che un AVL è bilanciato se questi ha, per ogni sotto-albero, un \emph{fattore di bilanciamento} \(B_{f}\) minore o uguale ad uno.
\begin{MarginNote}
    Qui per fattore di bilanciamento si intende la differenza in modulo tra l'altezza dei due sotto-alberi.
\end{MarginNote}

Per quel che riguarda le operazioni: essendo, come detto, che gli AVL sono dei BST, e poiché essa non modifica la struttura dell'albero,
la ricerca è analoga a quella dei BST; inserimento e cancellazione viceversa, proprio perché modificano la struttura dell'albero,
e rischiano di sbilanciarlo, sono modificate in modo tale che a seguito di esse l'albero risulti ancora bilanciato.
Tale modifica consiste nelle operazioni di rotazione descritte a seguito.

\subsubsection{Ribilanciamento di un AVL}
\subfile{../Livello III/Sottosottosezione 1.2.1 - Ribilanciamento di un AVL.tex}

\end{document}