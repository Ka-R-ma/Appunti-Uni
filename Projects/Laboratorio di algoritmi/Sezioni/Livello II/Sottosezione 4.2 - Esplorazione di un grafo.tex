\documentclass{subfiles}
\begin{document}
Dato \(G\) un grafo, esistono sostanzialmente di modi per visitarlo, queste sono
\begin{itemize}
    \item \emph{\textbf{visita (ricerca) in profondita (o DFS):}} scelto un nodo si esplora ciascun nodo a cui esso è connesso, fintanto che cio è possibile.
          Più precisamente, supposti \(u,v,z\) nodi del grafo, tali per cui \(\exists (u, v), (v, z) \in E\),
          visitando in DFS, si visita \(u\) lo si segna come visitato, da questi si passa ad \(v\), lo si visita, si segna come visitato e si passa ad \(z\).
          Si ripete la procedura per ogni nodo (non segnato come visitato) per cui esista un arco in \(E\).

    \item \emph{\textbf{visita in ampiezza (o BFS):}} differisce dalla DFS semplicemente perché, invece di visitare un singolo nodo per volta,
          visita in contemporanea tutti i nodi \(v\) connessi ad un nodo \(u\) per cui esiste un arco.
\end{itemize}
\end{document}