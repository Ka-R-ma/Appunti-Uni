\documentclass{subfiles}
\begin{document}
Ricordando il quicksort: si sceglie uno degli elementi da ordinare (il pivot) e ricorsivamente si suddivide l'insieme da ordinare in due sottoinsiemi,
ciascuno contenente rispettivamente i valori minori o uguali al pivot e quali maggiori ad esso.
3-way modifica la creazione di tali partizioni separando i valori minori da quelli uguali: si hanno dunque tre partizioni distinte \(R_{<}, R_{>}, R_{=}\).
Da un punto di vista implementativo, lo pseudo-codice è il seguente.
\subfile{../../Figure/Tikz Figure/Figure 5 - Pseudo codice 3-Way quicksort.tex}

\noindent Qui con \(\Abs{R}\) si indica il numero di stringhe rimaste.
\end{document}