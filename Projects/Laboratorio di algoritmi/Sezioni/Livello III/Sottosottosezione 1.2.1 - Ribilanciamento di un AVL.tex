\documentclass{subfiles}
\begin{document}
Alla base del processo di bilanciamento vi sono le operazioni di rotazione a sinistra e a destra.
Per comprendere tali operazioni, si faccia riferimento a \emph{Figura \ref{sub@Fig:1.a}}.
Si supponga di aggiungere ad \(T_{1}\) un nodo, portando così a uno sbilanciamento dell'albero (\emph{Figura \ref{sub@Fig:1.b}}).
In questo caso si dimostra sufficiente una singola rotazione a destra, a seguito della quale l'albero risulta bilanciato (\emph{Figura \ref{sub@Fig:1.c}}).
L'operazione appena descritta prende il nome di rotazione destra-destra,
a questa si aggiungono la rotazione sinistra-sinistra (simmetrica alla rotazione destra-destra) e le rotazioni sinistra-destra, destra-sinistra tra loro simmetriche.

\subfile{../../Figure/Tikz Figure/Figure 1 - Operazioni di rotazione degli AVL.tex}

Sia ora considerata l'operazione di rotazione sinistra-destra: prendendo in riferimento l'albero di \emph{Figura \ref{sub@Fig:1.a}},
si supponga di aggiungere al sotto-albero \(T_{2}\) un nodo; come evidente da \emph{Figura \ref{sub@Fig:2.a}} l'albero risulta ora sbilanciato.
\subfile{../../Figure/Tikz Figure/Figure 2 - Operazione di rotazione sinistra-destra.tex}
Poiché si osserva banalmente che una sola operazione di rotazione non è sufficiente; sia considerata la radice del sotto-albero, sia questa \(i\) (\emph{Figura \ref{sub@Fig:2.b}}),
in tal modo eseguendo dapprima una rotazione a sinistra (\emph{Figura \ref{sub@Fig:2.c}}) e successivamente una a destra (\emph{Figura \ref{sub@Fig:2.d}}),
l'albero risulta nuovamente bilanciato.



\end{document}