\documentclass{subfiles}
\begin{document}
Alla base del processo di bilanciamento vi sono le operazioni di rotazione a sinistra e a destra.
Per comprendere tali operazioni, si faccia riferimento a \emph{Figura \ref{sub@Fig:1.a}}.
Si supponga di aggiungere ad \(u\) un nodo, portando così a uno sbilanciamento dell'albero.
In questo caso si dimostra sufficiente una singola rotazione a destra, a seguito della quale l'albero risulta bilanciato.
L'operazione appena descritta prende il nome di rotazione destra-destra,
a questa si aggiungono la rotazione sinistra-sinistra (simmetrica alla rotazione destra-destra) e le rotazioni sinistra-destra, destra-sinistra tra loro simmetriche.

\subfile{../../Figure/Tikz Figure/Figure 1 - Operazioni di rotazione degli AVL.tex}
\begin{Remark*}
    Nel caso di rotazioni sinistra-sinistra (destra-destra equivalentemente) si dimostra sufficiente un'unica operazioni di rotazione elementare.
\end{Remark*}

Sia ora considerato il caso di una rotazione sinistra-destra (nel caso destra-sinistra si procede invertendo l'ordine delle operazioni),
per farlo si faccia riferimento a figura \emph{Figura \ref{sub@Fig:2.a}}.
Si supponga che ad \(y\) si aggiunga un nodo, sbilanciando in tal modo l'albero.
Come mostrato in \emph{Figura \ref{sub@Fig:2.b}}, una sola rotazione a sinistra non è sufficiente a bilanciare l'albero;
\subfile{../../Figure/Tikz Figure/Figure 2 - Operazione di rotazione sinistra-destra.tex}
a questa è infatti necessario aggiungere una rotazione a destra \emph{Figura \ref{sub@Fig:2.c}},
portando ad un AVL bilanciato come mostrato in \emph{Figura \ref{sub@Fig:2.d}}.
\end{document}