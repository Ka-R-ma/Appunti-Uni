\documentclass{subfiles}
\begin{document}
Un ternary search tree, è un trie in cui ogni nodo ha tre link: uno per sinistro che lo collega a stringhe di ordine inferiore,
uno destro che lo collega a stringhe di ordine superiore e un link centrale che indica una relazione di uguaglianza tra nodi.

Sempre considerando le stringhe \(\Set{pippo, pluto, topolino}\), il suo ternary tree è il seguente.
\subfile{../../Figure/Tikz Figure/Figure *.1 - Ternary search tree.tex}
\end{document}