\documentclass{subfiles}
\begin{document}
Sia \(G\) un grafo orientato. Si dirà che \(G\) ha un ordinamento topologico se esiste una permutazione dei suo nodi tale che,
per ogni arco della forma \((u, v) \in E\), si ha che \(u\) precede \(v\).
Si dimostra che se \(G\) è aciclico, allora esso ammette un ordinamento topologico. Valgono inoltre i seguenti teoremi.
\begin{Theorem}
    Sia \(G\) un grafo orientato e aciclico, allora la visita in \emph{reverse post-order} di \(G\) restituisce un ordinamento topologico.
\end{Theorem}
\begin{Theorem}
    Sia \(G\) un grafo diretto e privo di cicli, allora questi ha uno ordinamento topologico se e solo se \(\exists v \in V\) tale che questi non abbia archi entranti
    tale che \(G \setminus \Set{v}\) abbia un ordine topologico.
\end{Theorem}
\end{document}