\documentclass{subfiles}
\begin{document}
I \emph{practical algorithm to retrive information coded in alphanumerics tree} o più semplicemente PATRICIA tree, sono una versione compatta dei trie.
L'idea sostanziale è quella di non assegnare un arco ad ogni carattere della stringa, bensi utilizzare un'unico arco per prefissi comuni a più stringhe.
Con tale definizione, il trie di \emph{Figura \ref{Fig:8}}, si riduce a quanto mostrato in \emph{Figura \ref{Fig:10}}.
\subfile{../../Figure/Tikz Figure/Figure 10 - Example of PATRICIA tree.tex}
\end{document}