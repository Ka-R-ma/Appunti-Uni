\documentclass{subfiles}
\begin{document}
Un classificatore è detto lineare se, considerato il grafo di dispersione dei dati, questi sono distribuiti lungo una line retta.
Si distinguono ulteriormente due casi: \emph{semplice} se \(Y\) è dipendente da um'unica variabile \(X\), \emph{multipla} se dipendente da più \(X_{i}\).
Considerando il caso semplice, la relazione tra \(X \Text{e} Y\) è espressa, in genere, dalla seguente relazione
\[
    y = \alpha + \beta x + \varepsilon
\]
da cui stimare gli \(y_{i}\) equivale a calcolare
\[
    y_{i} = \alpha_{0} + \beta_{1} x_{i} + \varepsilon_{i}
\]
Tale calcolo, è in genere realizzato adoperando il metodo dei minimi quadrati, che risulta essere numericamente stabile.

\subsubsection{Metriche di regressione lineare}
\subfile{../Livello III/Sottosottosezione 7.1.1 - Metriche di regressione lineare.tex}
\end{document}