\documentclass{subfiles}
\begin{document}
Tipologia di addestramento in cui, sulla base dei dati di addestramento, si cerca di determinare la ``classe'' classe di appartenenza di nuovi dati.
In tale ``categoria'' rientrano algoritmi come
\begin{itemize}
    \item la \emph{regressione} con cui ai dati di input si assegna un valore numerico;
    \item la \emph{classificazione} grazie al quale i dati sono divisi sulla base della classe di appartenenza stimata.
\end{itemize}

\begin{Remark*}
    Tutto il processo di generalizzazione è di tipo statistico è basato sulla \emph{teoria di apprendimento statistico\footnotemark[2]}.
\end{Remark*}

Poiché la scelta della classe è determinata dalle conoscenze apprese, è opportuno che i seguenti errori risultino minimizzati.
\begin{itemize}
    \item \emph{Training error}, errore relativo ai dati di addestramento;
    \item \emph{test \emph{o} generalization error}, errore con cui si stima quanto correttamente il modello sia in grado di generalizzare.
          Nel particolare, fornendo al modello dati mai visti precedentemente, si verifica se questi sono correttamente categorizzati.
\end{itemize}
\end{document}