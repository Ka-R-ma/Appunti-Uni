\documentclass{subfiles}
\begin{document}
Si tratta di una tipologia di regressione che, in genere, è applicata quando il classificatore lineare non è sufficiente (\Eg{ dati non distribuiti linearmente}).
In questi casi allora, si procede a definire una nuova funzione \(f\), che di norma è un polinomio di grado superiore o una funzione trascendete,
cioè \(f\) è del tipo
\[
    f(x) = \alpha + \beta_{1} X_{1} + \cdots + \varepsilon
\]
di grado non superiore a quattro.

\subsubsection{Gradiant descent}
\subfile{../Livello III/Sottosottosezione 7.2.1 - Gradiant descent.tex}
\end{document}