\documentclass{subfiles}
\begin{document}
Nella precedente sezione si è fatto a delle trasformazioni applicabili ai dati.
Queste, atte a migliorare la qualità delle informazioni, si distinguono in
\begin{itemize}
    \item \emph{smoothing:} si tratta di una tecnica che applica un'ulteriore pulizia ai dati, così da rimuovere ulteriore rumore;
    \item \emph{aggregazione:} i dati sono combinati tra loro così da permettere la descrizione di concetti più generali;
    \item \emph{normalizzazione:} si riduce il range di valori assunti.
\end{itemize}
Per quanto riguarda la normalizzazione: esisto varie soluzioni, tra le più utilizzate: la \emph{scalatura decimale}, con la quale si riduce il range tra (0, 1);
e la \emph{z-core}, con la quale si sottrae ai dati la media degli stessi, e li si divide per la rispettiva deviazione standard.
\end{document}