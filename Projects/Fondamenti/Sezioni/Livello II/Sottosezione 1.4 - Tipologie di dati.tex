\documentclass{subfiles}
\begin{document}
I dati porterebbero essere suddivisi per diversi aspetti, è però di interesse la distinzione tra dati \emph{interdipendenti \emph{e} non-interdipendenti}.
Sostanzialmente la differenza è la seguente: nel caso di dati interdipendenti, questi hanno una qualche relazione  (\Eg{peso-altezza});
segue che informazioni in uno o più dati, dipendono o influenzano altri dati. Per i dati non-interdipendenti non si hanno tali relazioni.

\begin{Remark*}
    \`E giusto puntualizzare che il concetto di interdipendenza è fortemente legato al problema in esame.
\end{Remark*}
\end{document}