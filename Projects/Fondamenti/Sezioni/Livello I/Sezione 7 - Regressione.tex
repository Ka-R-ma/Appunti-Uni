\documentclass{subfiles}
\begin{document}
Il processo di regressione è interamente basato sul concetto di \emph{curve fitting}, per esso:
supposti \(X\) dati indipendenti, \(Y\) variabile dipendente è \(\beta\) parametri sconosciuti;
il curve fitting permette di definire una funzione tale che l'errore di adattamento ai dati sia minimo.

Parlando della regressione in se: questa è adoperata quando l'obbiettivo è quello di definire un modello che,
sulla base di variabili casuali \(\List{X}{1}{k}\), minimizzi il valore di un funzione dipendente \(Y\).

Si distinguono due tipologie di regressione: \emph{lineare \emph{e} non lineare}.

\subsection{Regressione lineare}
\subfile{../Livello II/Sottosezione 7.1 - Regressione lineare.tex}

\subsection{Regressione non lineare}
\subfile{../Livello II/Sottosezione 7.2 - Regressione non lineare.tex}
\end{document}