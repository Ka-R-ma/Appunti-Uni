\documentclass{subfiles}
\begin{document}
Quando si parla di clustering, si fa riferimento ad un insieme di algoritmi di apprendimento non-supervisionato,
principalmente applicati per operazioni di data mining. In generale si parla più precisamente di m-clustering, definito come segue.
\begin{Definition*}
    dato \(X \in \Real^{n}\) un vettore di features, si definisce \emph{m-clustering} di \(X\) il partizionamento di \(X\) in
    \(m\) classi \(\List{c}{1}{m}\), tali che
    \begin{itemize}
        \item \(c_{i} \ne \varnothing, i = 1, \ldots, m\);
        \item \(\Bigcup{C_{i}}{i = 1}[m] = X\);
        \item \(C_{i} \cap C_{j} = \varnothing, \forall i \ne j, i,j = 1, \ldots, m\).
    \end{itemize}
\end{Definition*}

Analizzando il processo di clustering nel dettaglio, questi può essere sintetizzato nelle seguenti fasi.
\begin{enumerate}
    \item \emph{quantizzazione delle similarità}: nel quale si definiscono i range entro i quali due diversi record appartengano ad una stessa classe;
    \item \emph{criterio di clustering}: si sceglie la funzione che determina le effettive classi;
    \item \emph{scelta dell'algoritmo di clustering};
    \item \emph{convalida dei risultati};
    \item \emph{interpretazione dei risultati}.
\end{enumerate}

\begin{Remark*}
    \`E opportuno puntualizzare che esistono diverse tipologie di clustering, tutte suddivisibili in \emph{partizionali \emph{o} gerarchici}.
\end{Remark*}

\subsection{Misure di qualità}
\subfile{../Livello II/Sottosezione 5.1 - Misure di qualita di clustering.tex}

\subsection{K-means}
\subfile{../Livello II/Sottosezione 5.2 - K-means.tex}
\clearpage
\end{document}