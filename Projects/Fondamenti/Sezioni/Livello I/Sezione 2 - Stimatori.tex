\documentclass{subfiles}
\begin{document}
\begin{Definition*}
    uno \emph{stimatore} (dal punto di vista statistico), è una funzione dei dati di cui si è in possesso,
    che fornisce la migliore approssimazione di una quantità/proprietà a cui si è interessati.
\end{Definition*}

Assunto che \(\Theta_{n}\) sia uno stimatore, \(\Theta\) il valore vero, si distinguono due casi
\begin{itemize}
    \item \(\Theta_{n}\) è polarizzato, cioè
          \[
              \Expected{\Theta_{n}} - \Theta \ne 0
          \]
          Da cui si deduce che lo stimatore commette un certo errore nell'approssimazione.

    \item \(\Theta_{n}\) non è polarizzato, da cui
          \[
              \Expected{\Theta_{n}} - \Theta = 0
          \]
          Si può inoltre dire che \(\Theta_{n}\) è asintoticamente non polarizzato se
          \[
              \Lim{n}{\infty}{\Expected{\Theta_{n}}} = \Theta
          \]
          o equivalentemente
          \[
              \Lim{n}{\infty}{\Expected{\Theta_{n}}} - \Theta = 0
          \]
\end{itemize}

In fine, supposto \(T_{n}\) uno stimatore, \(\tau\) il valore da stimare, si ha che
\[
    T_{n} \Text{è} \begin{cases}
        \Text{corretto} & \iff \Expected{T_{n}} = \tau          \\
        \Text{coerente} & \iff \Lim{n}{\infty}{\Var{T_{n}}} = 0
    \end{cases}\]

\subsection{Mean square error}
\subfile{../Livello II/Sottosezione 2.1 - MSE.tex}

\end{document}