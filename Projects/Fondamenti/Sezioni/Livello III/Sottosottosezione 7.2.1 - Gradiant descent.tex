\documentclass{subfiles}
\begin{document}
Si tratta di una tecnica che permette di minimizzare l'errore relativo alla funzione che definisce il regressore non lineare.
Si dimostra infatti che se si cerca di minimizzare suddetta funzione tramite curve fitting, ciò porta alla risoluzione di sistemi non lineari.
Per tale ragione, si utilizzano funzioni convesse per le quali si ha garanzia di convergenza.
Parlando del metodo in se, questi è un metodo iterativo che, ad ogni iterazione, minimizza l'errore sull'i-esimo passo.
\end{document}