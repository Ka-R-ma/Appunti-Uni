\documentclass{subfiles}
\begin{document}
Il criterio di massimizza verosimiglianza non sempre è applicabile, si procede in questi casi ad applicare il criterio di massima probabilità a posteriori.
Per esso, noto \(X = \Tuple{x}{1}{n}^{T}\) vettore di features, si deve calcolare \(\theta_{MAP}\) tale da massimizzare \(\Prob{\theta}[X]\).
Dal \emph{teorema di Bayes} si ha
\[
    \Prob{\theta}[X] = \Frac{\Prob{\theta}\Prob{X}[\theta]}{\Prob{X}}
\]
da cui segue che
\[\begin{aligned}
        \theta_{MAP} & = \arg \Max*{\Prob{\theta}[X]}[\theta]                               \\
                     & = \arg \Max*{\Frac{\Prob{\theta}\Prob{X}[\theta]}{\Prob{X}}}[\theta]
    \end{aligned}\]
\end{document}