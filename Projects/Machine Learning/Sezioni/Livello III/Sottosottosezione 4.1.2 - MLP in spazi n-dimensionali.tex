\documentclass{subfiles}
\begin{document}
Da quanto detto, i MLP permettono di effettuare riconoscimenti in spazi bi-dimensionali.
Ci si chiede se è possibile utilizzarli in spazi n-dimensionali.
Si dimostra che tale generalizzazione è possibile se e solo se si definisce opportunamente il MLP.
\begin{Example*}
    Sia supposto di dover definire un MLP, tale che questi permetta di stabilire se un dato punti sia nella regione pentagonale nella figura a seguire.
    \subfile{../../Tikz/Figure *.6 - MLP n-dimensionale.tex}
    Osservando che quanto richiesto può essere espresso come un funzione booleana a cinque variabili,
    questa può essere approssimata da un MLP a due livelli come nella figura a seguire.
    \subfile{../../Tikz/Figure *.7 - MLP n-dimensionale 2.tex}
\end{Example*}
\end{document}