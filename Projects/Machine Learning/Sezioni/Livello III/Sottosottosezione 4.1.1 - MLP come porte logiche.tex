\documentclass{subfiles}
\begin{document}
Come anticipato, le MLP possono essere impiegate per realizzare una qualsiasi porta logica.
\begin{Example*}
    si supponga di dover realizzare una porta NOT, la cui tavola di verità si ricorda essere la seguente.
    \subfile{../../Tikz/Figure *.4 - Tavola della verita NOT.tex}

    Segue banalmente che una rappresentazione tramite MLP, è descritta dalla seguente struttura
    \subfile{../../Tikz/Figure *.5 - NOT tramite MLP.tex}
    Qui assunto \(A\) l'input, resta da definire \(b\) bias e \(f(x)\).
    Segue però da una semplice osservazione che, una soluzione valida è \(b = 1, f(x) = -A + b\).
\end{Example*}
\end{document}