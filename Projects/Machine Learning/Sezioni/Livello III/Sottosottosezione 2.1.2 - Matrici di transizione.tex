\documentclass{subfiles}
\begin{document}
Sia considerata \emph{Equazione \ref{Eq:2}}, si osserva che questa rappresenta il prodotto riga-colonna di una qualche matrice.
Sia definita allora
\[
    H(k, k + n) = \begin{bmatrix}
        p_{ij}(k, k + n)
    \end{bmatrix}
\]
come la matrice di transizione a n passi.
Con tale formulazione, l'equazione di Chapman-Kolmogorov, può essere riscritta equivalentemente come
\[
    H(k, k + n) = H(k, u)H(u, k + n)
\]
da cui scegliendo opportunamente \(u\) è possibile definire
\begin{itemize}
    \item \emph{l'equazione di Chapman-Kolmogorov in avanti} se si pone \(u = k + n - 1\), da cui
          \[\begin{aligned}
                  H(k, k + n) & = H(k, k + n - 1)H(H + n - 1, k + n) \\
                              & = H(k, k + n - 1)P(k + n - 1)
              \end{aligned}\]
          oppure;
    \item \emph{l'equazione di Chapman-Kolmogorov all'indietro} se si pone \(u = k + 1\), da cui
          \[\begin{aligned}
                  H(k, k + n) & = H(k, k + 1)H(H + 1, k + n) \\
                              & = P(k)H(k + 1, k + n)
              \end{aligned}\]
          con \(P\) \emph{matrice di transizione}.
\end{itemize}
Inoltre, se \(P(k) = P\), ossia
\[
    p_{ij}(k) = \Prob{X_{k + 1} = j}[X_{k} = i] = \Prob{X_{k} = j}[X_{k - 1} = i]
\]
allora il processo di Markov sarà detto \emph{omogeneo}.
\end{document}