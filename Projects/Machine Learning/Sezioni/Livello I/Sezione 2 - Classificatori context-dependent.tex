\documentclass{subfiles}
\begin{document}
I classificatori discussi sinora, sono utilizzabili in casi in cui i dati sono simultaneamente presenti,
ma soprattutto se le classi sono unicamente dipendenti dai valori assunte dalle stesse e da quello delle features.
Esistono però scenari in cui tale situazione non si verifica; è pertanto necessario poter definire modelli che apprendono dinamicamente.

\subsection{Classificatori Bayesiani context-dependent}
\subfile{../Livello II/Sottosezione 2.1 - Classificatori Bayesiani context-dependent.tex}

\subsection{Classificatori Bayesiani pt.2}
\subfile{../Livello II/Sottosezione 2.2 - Classificatori Bayesiani pt2.tex}

\subsection{Addestramento e riconoscimento attraverso HMM}
\subfile{../Livello II/Sottosezione 2.3 - Addestramento e riconoscimente attraverso HMM.tex}
\end{document}