\documentclass{subfiles}
\begin{document}
Si parta dal considerare il caso bidimensionale.
Sia \(X = \Tuple{x}{1}{n}^{T}\) un vettore di features, e siano \(\omega_{1}, \omega_{2}\) le due possibili classi.
Allora se rappresentato \(X\) su di un piano, è possibile identificare due regioni, siano queste \(R_{1}, R_{2}\), tali che
\[X \in \begin{cases}
        R_{1} \iff \Prob{\omega_{1}}[X] > \Prob{\omega_{2}}[X], \\
        R_{2} \iff \Prob{\omega_{2}}[X] > \Prob{\omega_{1}}[X].
    \end{cases}\]
Si definisce superficie decisionale una funzione \(g(x)\) tale che \(\Prob{\omega_{1}}[X] - \Prob{\omega_{2}}[X] = 0\).

Più in generale, supposto \(X = \Tuple{x}{1}{n}^{T}\) un vettore di features e \(\omega = \Tuple{\omega}{1}{m}\) possibili classi,
una superficie decisionale è una funzione \(g_{ij}(x)\) tale che \(\Prob{\omega_{i}}[X] - \Prob{\omega_{j}}[X] = 0\), \(\forall i, j \in \Set{1, \ldots, m}, i \ne j\).
\end{document}