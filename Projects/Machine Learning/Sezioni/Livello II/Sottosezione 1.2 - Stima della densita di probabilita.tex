\documentclass{subfiles}
\begin{document}
Noto come calcolare la probabilità per ogni classe, resta il problema di come identificare la distribuzione di probabilità dei dati.
Si distinguono in questo contesto due approcci:
\begin{itemize}
    \item \emph{approccio parametrico:} è nota la forma funzionale dei dati, da cui è facile ricavare la distribuzione di probabilità;
    \item \emph{approccio non-parametrico:} sono noti i valori di alcune features, si può allora stimare la forma funzionale.
\end{itemize}
Nello specifico a seguito ci si concentra suglia approcci funzionali,
in particolare saranno trattati i criteri di \emph{massima verosimiglianza \emph{e} massima probabilità a posteriori}.

\subsubsection{Massima verosimiglianza}
\subfile{../Livello III/Sottosottosezione 1.2.1 - Citerio di massimizza verosimiglianza.tex}

\subsubsection{Massima probabilità a posteriori}
\subfile{../Livello III/Sottosottosezione 1.2.2 - Citerio di massima probabilita a posteriori.tex}
\end{document}