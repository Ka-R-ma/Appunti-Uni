\documentclass{subfiles}
\begin{document}
Sia considerato un modello i cui stati non sono direttamente osservabili, ma si può unicamente risalire ad essi solo dai dati di addestramento:
tale tipologia di modelli sono detti \emph{hidden Markov models}.

Tali modelli sono in genere applicati per una serie di problemi di natura bio-informatica.
Più in generale, gli HMM sono descritti dalla seguente quadrupla
\[
    S = (\Prob{i}[j], \Prob{X}[j], \Prob{i}, k)
\]
con
\begin{itemize}
    \item \(\Prob{i}[j]\) insieme delle probabilità di transizione;
    \item \(\Prob{X}[i]\) insieme delle probabilità a priori;
    \item \(\Prob{i}\) insieme delle probabilità di stato iniziale;
    \item \(k\) numero degli stati.
\end{itemize}

Considerando ora un'applicazione di tali modelli, si descrive a seguito il \emph{pattern recognition}.

\subsubsection{Pattern recognition}
\subfile{../Livello III/Sottosottosezione 2.3.1 - Pattern recognition.tex}
\end{document}