\documentclass{subfiles}
\begin{document}
Come detto in generale i classificatori bayesiani operano bene in molti casi.
Vi sono casi però in cui è necessario calcolare le probabilità congiunte in maniera esatta, nascono per tale ragione le reti Bayesiane.
Queste procedono come semplici classificatori bayesiani, ma all'occorrenza calcolano opportunamente le probabilità congiunte.
\begin{Example*}
    sia supposto \(X = \Tuple{x}{1}{4}\) secondo le relazioni rappresentate dal grafo a seguito riportato.
    \subfile{../../Tikz/Figure *.1 - Esempio di rete bayesiana.tex}

    \noindent e siano \(c = \Tuple{c}{1}{3}\) classi, allora
    \[
        \Prob{c_{i}}[X] = \Prob{c_{i}} \Prob{x_{3}}[x_{2}x_{1}] \Prob{x_{4}}
    \]
    Più in generale, una rete (o network) bayesiana è un grafo diretto e aciclico i cui nodi rappresentano le variabile.
\end{Example*}
\end{document}