\documentclass{subfiles}
\begin{document}
Si parta dal considerare i diversi stili di apprendimento umano, questi risultano essere: \emph{procedurali, \emph{di} classificazione \emph{e di} memorizzazione}.
Nel caso del machine learning, i tipi di apprendimento di interesse risultano essere quelli procedurali e di classificazione.

Partendo dal considerare quest'ultimi, obbiettivo è quello di definire un modello che, basandosi sui dati di addestramento,
sia capace di classificare correttamente dati nuovi. Considerando invece modelli di tipo procedurale, a partire da ``ambienti'' contenti ``ricompense'',
definisco un modello che sia capace di prendere decisioni.
In quest'ultimo caso si distinguono due scenari
\begin{itemize}
    \item gli ``ambienti'' sono noti, allora si può utilizzare la programmazione dinamica per valutare l'iterazione e agire di conseguenza,
    \item gli ``ambienti'' sono ignoti, si utilizza il reinforced learning, agendo e osservando l'ambiente.
\end{itemize}

\subsubsection{Apprendimento rinforzato}
\subfile{../Livello III/Sottosottosezione 2.4.1 - Apprendimento rinforzato.tex}

\subsubsection{Processi decisionali di Markov}
\subfile{../Livello III/Sottosottosezione 2.4.2 - Processi decisionali di Markov.tex}
\end{document}