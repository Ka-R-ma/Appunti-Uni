\documentclass{subfiles}
\begin{document}
La necessità di semplificare la scrittura di codice sorgente, porta alla nascita dei primi linguaggi di programmazione.
Ciò preclude quindi un ``meccanismo'' che permetta di descrivere il come e il cosa, si possa fare con un dato linguaggio.
E, segue banalmente, che tale meccanismo può essere definito solo con un linguaggio esistente.
Per questa e altre ragioni, nascono i compilatori e gli interpreti.

\begin{Note*}
    sebbene non di interesse ai fini del corso, a seguito si fa una breve digressione sugli interpreti.
\end{Note*}
A partire dal codice sorgente, un interprete converte, istruzione per istruzione, il sorgente che è immediatamente eseguito.
Linguaggi di questo tipo sono \emph{python, perl, ecc}.
\\ \\
Parlando ora dei compilatori, questi convertono il source-code in un codice macchina \emph{\textbf{equivalente}}.
Ulteriore compito dei compilatori è quello di segnalare eventuali errori.

\begin{Note*}
    esistono linguaggi (eg. JAVA) che fanno uso di compilatori ibridi: ossia compilatori che implementano sia la compilazione,
    sia l'interpretazione del sorgente.
\end{Note*}

\subsection{Struttura di un compilatore}
\subfile{../Livello II/Sottosezione 1.1 - Struttura di un compilatore.tex}
\clearpage
\end{document}