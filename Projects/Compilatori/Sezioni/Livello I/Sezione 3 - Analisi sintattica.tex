\documentclass{subfiles}
\begin{document}
Secondo step della compilazione è l'analisi sintattica, con la quale si verifica se il programma rispetta le regole sintattiche del linguaggio.
Tali regole sono definite attraverso grammatiche context-free.

\begin{Note*}
    come per le espressioni regolari, si presuppongono conoscenze assodate sulle CFG.
    Per tale ragione non saranno trattate.
\end{Note*}
Definita la sintassi del linguaggio, è compito del parser verificare che ciascuno dei token generati dal lexer, possa effettivamente essere generato.

Parlando dei parser ,se ne distinguono tre classi: top-down. i bottom-up e gli universali.
A quest'ultima categoria appartengono gli algoritmi di \emph{Cocke-Younger-Kasami} e quello di \emph{Earley} a seguito descritto.
Entrambi gli algoritmi appena citati sono, come tutti i parser universali, capaci di riconoscere qualsiasi CFG,
ma per tale ragione risultano troppo inefficienti per scopi pratici.

\subsection{Algoritmo di Earley}
\subfile{../Livello II/Sottosezione 3.1 - Algoritmo di Earley.tex}
\clearpage

\subsection{Gestione degli errori in un parser}
\subfile{../Livello II/Sottosezione 3.2 - Gestione degli errori in un parser.tex}

\subsection{Top-Down parser}
\subfile{../Livello II/Sottosezione 3.3 - TopDown parse.tex}
\clearpage

\subsection{Parser LL(1)}
\subfile{../Livello II/Sottosezione 3.4 - Parser LL(1).tex}
\clearpage

\subsection{Parser Bottom-Up}
\subfile{../Livello II/Sottosezione 3.5 - Parser BottomUp.tex}
\end{document}