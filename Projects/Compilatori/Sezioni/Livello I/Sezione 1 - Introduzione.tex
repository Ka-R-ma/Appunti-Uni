\documentclass{subfiles}
\begin{document}
Con lo svilupparsi dei linguaggi di programmazione, si sono sviluppati parallelamente gli \emph{interpreti \emph{e i} compilatori}.
Questi ultimi, la cui struttura principale è mostrata in \emph{Figura \ref{fig:1}},
\subfile{../../Figure/Tikz Figure/Figure 1 - Struttura dei compilatori.tex}
permettono di descrivere il come e il cosa si possa fare con il linguaggio che essi definiscono.
Nello specifico, un compilatore converte il codice sorgente in un codice macchina \emph{\textbf{equivalente}},
in aggiunta al segnalare eventuali errori.
Per quel che riguarda gli interpreti, questi convertono istruzione per istruzione il sorgente e lo eseguono immediatamente.
Tra i linguaggi di questo tipo: \emph{python, perl, ecc.}

\begin{Remark*}
    La struttura di \emph{Figura \ref{fig:1}}, è limitata alle fasi di interesse del corso.
\end{Remark*}

\subsection{Richiami alle RegEx e alle grammatiche}
\subfile{../Livello II/Sottosezione 1.1 - Richiami alle espressioni regolari e alle grammatiche.tex}
\clearpage
\end{document}