\documentclass{subfiles}
\begin{document}
\begin{Definition*}
    sia \(G\) una grammatica, questa è detta essere LL(1) se e solo se \(\forall \Grammar{A}{\alpha}[\beta]\) si ha che:
    \begin{itemize}
        \item \(\alpha \text{e} \beta\) non derivano stringhe che iniziano con uno stesso terminale \(a\).
              Cioè \(FIRST(\alpha) \neq FIRST(\beta)\).
        \item al più uno dei due deriva \(\varepsilon\).
        \item se \(\beta \GrammarDerivation \varepsilon\), allora \(\alpha\) non deriva stringhe che iniziano con terminali in \(FOLLOW(A)\). O viceversa.
    \end{itemize}
\end{Definition*}
\begin{Algorithm*}[di parsing per LL(1)]
    Dalla definizione segue la possibilità di realizzare un parser predittivo per le grammatiche LL(1), il cui pseudo-codice è di seguito riportato.
    \begin{center}
        \begin{lstlisting}
            LL_1_parser(stack p, input i, LL_1_table M, axiom S):
                error = false;
                p.push(\textdollar);
                p.push(S);
                while ()__* p.top() \(\neq\) \textdollar*__ AND i.top() __* \textdollar*__ AND !error) do {
                    if isTerminal(p.top()){
                        if p.top() == i.top(){
                            p.pop();
                            i.next();
                        }
                        error = true;
                    }
                    if isEmpty(M[p.top(), i.top()])
                        error = true;
                    else{
                        p.pop();
                        for (j = n; j > 0; j--)
                            p.push(__* \(X_{j}\) *__)
                    }
                }
                if (!error)
                    accept();
                raise_error();
        \end{lstlisting}
    \end{center}
\end{Algorithm*}

\end{document}