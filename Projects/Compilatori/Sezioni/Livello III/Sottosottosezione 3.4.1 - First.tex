\documentclass{subfiles}
\begin{document}
Come detto si tratta di una delle funzioni utilizzate per la costruzione della tabella.
Nello specifico, assunto una stringa $\alpha \in N \times T\footnotemark[1]$, si ha che
$$
    FIRST(\alpha) = \Set{x \in T}[\alpha \Derivation x\beta]
$$
In altri termini, $FIRST(\alpha)$ rappresenta l'insieme dei terminali tali che gli stessi compaiano come primo termine di una qualche derivazione di $\alpha$.
Inoltre se $\alpha \Derivation \varepsilon$ allora $\varepsilon \in FIRST(\alpha)$.
Per quel che concerne l'algoritmo per l'implementazione di FIRST, questi è il seguente.
\subfile{../../Figure/Other/Figure 3 - Pseudo-codice procedura FIRST.tex}
\footnotetext[1]{Rispettivamente, insieme dei non-terminali e dei terminali.}
\end{document}