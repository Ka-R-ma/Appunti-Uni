\documentclass{subfiles}
\begin{document}
Come detto si tratta di una delle funzioni utilizzate per la costruzione della tabella.
Nello specifico, assunto una stringa \(\alpha \in N \times T\footnotemark[1]\), si ha che
\[
    FIRST(\alpha) = \Set{x \in T}[\alpha \Derivation x\beta]
\]
In altri termini, \(FIRST(\alpha)\) rappresenta l'insieme dei terminali tali che gli stessi compaiano come primo termine di una qualche derivazione di \(\alpha\).
Inoltre se \(\alpha \Derivation \varepsilon\) allora \(\varepsilon \in FIRST(\alpha)\).

\begin{Algorithm*}
    Per quel che concerne l'algoritmo per l'implementazione di FIRST, questi è il seguente.
    \begin{center}
        \begin{lstlisting}[language = PSEUDO]
            for all terminal __*\(x\)*__ and __*\(\varepsilon\)*__ do
                __*FIRST(x) = \(\Set{x}\)*__;
            for all non-terminal __*\(X\)*__ do
                __*FIRST(X) = \(\varnothing\)*__;
            while there are changes in any FIRST do
                for each __*\( \Grammar{X}{\List{Y}{1}{k}}\)*__ do {
                    i = 1;
                    continue = true;
                    while continue == true AND i <= k do {
                        add __*FIRST(\(Y_{i}\)) \(\setminus \Set{\varepsilon}\)*__ to FIRST(X);
                        if __*\(\varepsilon \notin FIRST(Y_{i})\)*__
                            continue = false;
                        i += 1;
                    }
                    if continue == true
                        add __*\(\varepsilon\)*__ to FIRST(X);
                 }
        \end{lstlisting}
    \end{center}
\end{Algorithm*}
\footnotetext[1]{Rispettivamente, insieme dei non-terminali e dei terminali.}
\end{document}