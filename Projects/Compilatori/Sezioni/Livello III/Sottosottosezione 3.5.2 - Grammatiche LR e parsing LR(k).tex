\documentclass{subfiles}
\begin{document}
Il parser shif-reduce descritto precedentemente, potrebbe non essere sufficiente a riconoscere alcune grammatiche.
Per tale ragione, in generale, si preferisce utilizzare i parser LR(k), con ``k'' numero di caratteri successivi di cui tenere traccia ad ogni passo.

\begin{Note*}
    Saranno trattati nel dettaglio i parser LR(0). ALtri parser LR (\textbf{eg:} SLR e/o LALR(1)) saranno solamente accennati.
\end{Note*}

\begin{Remark*}
    Se una grammatica è ambigua, non è LR(k) per nessun k.
\end{Remark*}

\paragraph{Costruzione della tabella LR(0)}
\subfile{../Livello IIII/Sottosottosottosezione 3.5.2.1 - Costruzione della tabella LR(0).tex}
\end{document}