\documentclass{subfiles}
\begin{document}
I parser LR(1) sono un'estensione dei parser LR(0); nello specifico permettono di risolvere conflitti reduce/reduce a costo di un numero molto maggiore di item.

\begin{Remark*}
    Proprio perché con LR(1) il numero di item tende a crescere rapidamente, nella pratica si utilizza una versione ``ottimizzata'' nota come LALR(1)
\end{Remark*}

Circa la costruzione della tabella di parsing, questa è analoga alla costruzione dell'LR(0) con le seguenti differenze:
\begin{itemize}
    \item gli item sono della forma \((\Grammar{A}{\alpha . \beta}, t)\),
          con \(\Grammar{A}{\alpha . \beta}\) item LR(0) e \(t\) simbolo (o sequenza di simboli nel caso di LR(k)) di lookahead;

    \item le operazioni di reduce \(\Grammar{X}{\gamma}\) si effettuano se e solo se \(\exists (\Grammar{X}{\gamma.}, t)\).
\end{itemize}

Parlando ora di LALR, questi è un'ottimizzazione di LR(1), ottimizzazione che segue da un'osservazione: se si ignorano i simboli di lookahead,
molti stati presentano item equivalenti (il cosiddetto \emph{core}); se dunque tali stati presentano simboli di lookahead distinti (le cosiddette \emph{prospezioni}),
si può pensare di aggregare questi stati in un unico stato equivalente, procedendo a opportune ridenominazioni delle transizioni.
\end{document}