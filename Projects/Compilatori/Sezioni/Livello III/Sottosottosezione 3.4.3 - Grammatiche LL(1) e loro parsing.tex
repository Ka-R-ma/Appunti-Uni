\documentclass{subfiles}
\begin{document}
\begin{Definition*}
    sia $G$ una grammatica, questa è LL(1) se e solo se $\forall \Grammar{A}{\alpha}[\beta]$ si ha che:
    \begin{itemize}
        \item $\alpha \text{e} \beta$ non derivano stringhe che iniziano con uno stesso terminale $a$.
              Cioè $FIRST(\alpha) \neq FIRST(\beta)$.
        \item al più uno dei due deriva $\varepsilon$.
        \item se $\beta \Derivation \varepsilon$, allora $\alpha$ non deriva stringhe che iniziano con terminali in $FOLLOW(A)$, o viceversa.
    \end{itemize}
\end{Definition*}

Dalla definizione segue la possibilità di realizzare un parser predittivo per le grammatiche LL(1), il cui pseudo-codice è di seguito riportato.
\subfile{../../Figure/Other/Figure 5 - Pseudo-codice parser LL1.tex}

\noindent Per struttura delle grammatiche LL(1) e da un'analisi dell'algoritmo, si osserva che il parser LL(1) ha complessità \OrderOf{n},
con $n$ lunghezza dell'input.

\end{document}