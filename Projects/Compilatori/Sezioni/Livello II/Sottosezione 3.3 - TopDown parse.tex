\documentclass{subfiles}
\begin{document}
Come suggerito dal nome, si tratta di parser che verificano la generabilità dell'input a partire dall'assioma.
Si osservi però che tali parser soffrono di un problema: non sono deterministici.
Banalmente, motivo di ciò è dato dal fatto che un non terminale possa produrre delle derivazioni, nel seguito indicate come \(\Derivation\),
che iniziano con uno stesso carattere.
\\ \\
Prima di descrivere il principale dei parser top-down, si descrivono due tecniche in genere utilizzate per eliminare il non determinismo.
\begin{enumerate}
    \item Si supponga il caso di un non terminale che ha un prefisso comune a diversi terminali.
          Cioè si ha qualcosa del tipo
          \[
              \Grammar{A}{\gamma \alpha_{1}}[\gamma \alpha_{2}, \cdots, \gamma \alpha_{n}, \omega]
          \]
          con \(\gamma \in T, \alpha_{i} \in N \cup T, \forall i = 0, \ldots, n\). T insieme di terminali, N dei non terminali.

          Per risolver il problema si introduce un nuovo non terminale, così da posticipare la scelta.
          Cioè, la grammatica di cui sopra diventa
          \[\begin{aligned}
                   & \Grammar{A}{\gamma B}[\gamma B, \cdots, \gamma B, \omega] \\
                   & \Grammar{B}{\gamma \alpha_{1}}[\cdots, \alpha_{n}]        \\
              \end{aligned}\]
    \item Si supponga ora che un non terminale presenti una ricorsione sinistra.
          Si ha cioè qualcosa del tipo
          \[
              \Grammar{A}{A \alpha_{1}}[A \alpha_{2}, \beta_{1}, \beta_{2}]
          \]
          considerando unicamente il caso in cui la ricorsione sia immediatamente a sinistra,
          come prima si introduce un nuovo non terminale cosi da ritardare la scelta.
          Dalla grammatica di sopra si ottiene dunque qualcosa del tipo
          \[\begin{aligned}
                   & \Grammar{A}{\beta_{1} B}[\beta_{2} B]                \\
                   & \Grammar{B}{\alpha_{1} B}[\alpha_{2} B, \varepsilon] \\
              \end{aligned}\]
\end{enumerate}
Circa i parser top-down in se, si distinguono
\begin{itemize}
    \item i parser \emph{ a discesa ricorsiva}: di poco interesse al corso;
    \item i parser \emph{LL(k)}: sono parser che analizzano l'input da sinistra a destra,
          costruendo una derivazione sinistra sulla base dei k simboli successivi.
          \begin{Note*}
              saranno considerati parser LL(1).
          \end{Note*}
\end{itemize}

\end{document}