\documentclass{subfiles}
\begin{document}
Lo scanner analizza il sorgente carattere per carattere, ma poiché un token potrebbe essere composto da più caratteri,
è necessario un metodo di ``backtracking'': cioè un modo per poter tenere traccia di dove un token inizi.
Ciò è generalmente realizzato con un doppio buffering. Con l'ausilio di due puntatori, \lstinline{forward} e \lstinline{lexem_begin},
si procede ad identificare i token. Nello specifico inizialmente i due puntatori coincidono,
successivamente si fa avanzare \lstinline{forward} fintantoché si riscontra un lessema. Fatto ciò si aggiorna la posizione di \lstinline{lexem_begin}.
\end{document}