\documentclass{subfiles}
\begin{document}
I parser LR(0) possono essere intesi come una versione migliorata dei parser SR.
Partendo dalla costruzione della tabella: similarmente ai parser LL(1), per indicare a che punto una produzione è stata analizzata,
si aggiunge in quella posizione un punto.

\begin{Example*}
    si supponga la seguente produzione: \(\Grammar{A}{XYZ}\);
    in questo caso le posizioni distinte in cui è possibile trovarsi sono
    \[
        \Grammar{A}{.XYZ} \quad \Grammar{A}{X.YZ} \quad \Grammar{A}{XY.Z} \quad \Grammar{A}{XYZ.}
    \]
    ognuna delle ``produzioni'' di cui sopra è detta \emph{item}.
\end{Example*}

\subsubsection{Costruzione dell'automa}
\subfile{../Livello III/Sottosottosezione 3.7.1 - Costruzione dell'automa LR0.tex}

\subsubsection{Problematiche di LR(0)}
\subfile{../Livello III/Sottosottosezione 3.7.2 - Problemi di LR(0).tex}

\subsubsection{Esempio di parsing LR(0)}
\subfile{../Livello III/Sottosottosezione 3.7.3 - Esempio di parsing LR0.tex}
\clearpage
\end{document}