\documentclass{subfiles}
\begin{document}
Come anticipato, i parser LL(k) sono parser che analizzano l'input da sinistra verso destra,
effettuando derivazioni sinistre sulla base dei k simboli successivi.

Considerando il parser in se, questi deve la propria efficacia alla costruzione di una matrice/tabella, strutturata come segue:
\begin{itemize}
    \item ogni carattere non-terminale rappresenta una riga della tabella, i terminali le colonne;
    \item a partire dall'assioma, ogni regola del tipo \(\Grammar{X}{\gamma}\) tale per cui \(\gamma \Derivation tB\),
          è inserita in posizione \((X, t)\);

    \item ogni regola del tipo \(\Grammar{X}{\gamma}\) tale che \(\gamma \Derivation \varepsilon\),
          è inserita in \((X, t), \forall t \Such S \Derivation \beta X t a\).
\end{itemize}

\subsubsection{FIRST}
\subfile{../Livello III/Sottosottosezione 3.4.1 - First.tex}

\subsubsection{FOLLOW}
\subfile{../Livello III/Sottosottosezione 3.4.2 - Follow.tex}
\clearpage
\subsubsection{Grmmatiche LL(1) e loro parsing}
\subfile{../Livello III/Sottosottosezione 3.4.3 - Grammatiche LL(1) e loro parsing.tex}

\subsubsection{Costruzione della tabella}
\subfile{../Livello III/Sottosottosezione 3.4.4 - Costruzione della tabella.tex}
\clearpage
\end{document}