\documentclass{subfiles}
\begin{document}
Come detto Earley accetta qualsiasi CFG, più nello specifico: data \(x_{1} \ldots x_{n}\) una stringa,
scandendo la stessa da sinistra a destra, per ogni \(x_{i}\) si costruiscono stati \(S_{j}\), stati del tipo \lstinline{(dotted_rule, address)}.
Qui \lstinline{dotted_rule} sta ad indicare una produzione della grammatica alla cui destra è posto un punto,
per per tenere traccia della posizione della ``sotto-stringa'' esaminata. Con \lstinline{address} si indica invece la posizione del punto.

\begin{Example*}
    si supponga uno stato \((A \to \alpha . \beta, i)\). Ciò sta ad indicare che si è esaminata la sola sotto-stringa \(\alpha\).
\end{Example*}

\begin{Algorithm*}[di Earley]\noindent
    \begin{lstlisting}[language = PSEUDO]
        input = __*$x_{1} \ldots x_{n}$*__
        __*$x_{n + 1}$*__ = __*\textdollar*__
        for j = 0 to n do
            foreach state in S[j] choose between: scanner, predictor, completed
        if S[n + 1] == __*\textdollar*__
            accept()
        refuse()
    \end{lstlisting}
\end{Algorithm*}

\subsubsection{Scanner}
\subfile{../Livello III/Sottosottosezione 3.1.1 - Scanner.tex}

\subsubsection{Predictor}
\subfile{../Livello III/Sottosottosezione 3.1.2 - Predictor.tex}

\subsubsection{Completed}
\subfile{../Livello III/Sottosottosezione 3.1.3 - Completed.tex}

\subsubsection{Complessità di Earley}
\subfile{../Livello III/Sottosottosezione 3.1.4 - Complessita di Earley.tex}
\end{document}